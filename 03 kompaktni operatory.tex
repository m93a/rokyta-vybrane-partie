\section{Kompaktní operátory}

\begin{definition}[Kompaktní operátor]
Nechť $X,Y$ jsou Banachovy prostory, $\map T: X \mapsto Y$ je lineární operátor. Řekneme, že $\map K$ je kompaktní, jestliže pro každou omezenou množinu $A \subset X$ platí, že $\overline{\map K(A)} \subset Y$ je kompatkní množina. Množinu všech kompaktních operátorů zapisujeme jako $\comp(X,Y)$ a zavádíme $\comp(X) := \comp(X,X)$.
\end{definition}

\begin{remark}
\begin{enumerate}
    \item Platí, že $\comp(X,Y) \subset \lin(X,Y)$. Je-li totiž $A \subset X$ omezená, pak $\overline{\map K(A)} \subset Y$ je kompaktní a podle nutné podmínky kompaktnosti musí být $\overline{\map T(A)} $ omezená a uzavřená množina.
    \item Připomeňme, že omezenost a uzavřenost nejsou v nekonečné dimenzi postačující podmínky pro kompaktnost. V konečné dimenzi tomu tak ovšem je.
\end{enumerate}
\end{remark}
Pro kompaktní množiny můžeme používat weierstrassovské vybírání podposloupnosti.
\begin{theorem}[O charakterizaci kompaktního operátoru pomocí posloupnosti]
Operátor $\map K :X \mapsto Y$ je kompaktní právě tehdy, když pro každou omezenou posloupnost $\sequence {x_n}{n} \subset X$ existuje vybraná podposloupnost $\sequence{x_{n_k}}{k} $ a prvek $y \in Y$ takové, že $\map K ( x_{n_k}) \rightarrow y $.
\end{theorem}

V následujícím budeme diskutovat tuto úvahu: pokud by celý prostor $Y$ měl vlastnost, že z každé omezené posloupnosti v $Y$ se dá vybrat konvergentní podposloupnost, pak by platilo \begin{align*}
    \lin(X,Y) = \comp (X,Y) \:.
\end{align*}
Tuto úvahu zdůvodníme následovně. Stačí ukázat, že 
\begin{align*}
    \lin(X,Y) \subset \comp (X,Y) \:,
\end{align*}
opačnou implikaci jsme již vyřešili. Buď tedy $\map T \in \lin(X,Y)$ a $\sequence{x_n}{n} \subset X$ omezená posloupnost.
Díky spojitosti $\map T$ je posloupnost $ \sequence{ \map T x_n}{n}$ omezená. Pokud bychom zaručili, že z každé takové posloupnosti už můžeme vybrat konvergentní podposloupnost, dokázali bychom tím kompaktnost každého spojitého operátoru. Takovou vlastnost jistě nemůže mít každý metrický prostor. Na počest Bolzanovy-Weierstrassovy věty platné v $\R$ ji nazveme \uv{B-W vlastností}.

\begin{lemma}[O nutné podmínce pro B-W vlastnost]
Nechť $Y$ je Banachův prostor. Pak $Y$ má B-W vlastnost právě tehdy, když má konečnou dimenzi.
\end{lemma}
\begin{proof}[Náznak důkazu:]
\uv{$\Leftarrow$} Na $\R$ můžeme používat Bolzanovu-Weierstrassovu větu z prvního semestru. V prostoru $\R^N$ můžeme provést postupné výběry po složkách. Nyní v každém konečnědimenzionálním prostoru $Y$ můžeme najít bázi a každému prvku $y \in Y$ přiřadit $n$-tici souřadnic vzhledem k této bázi. 

\uv{$\Rightarrow$} Nechť je $Y$ nekonečnědimenzionální. Zvolme $y_1 \in Y$ a poté indukcí vybíráme prvky $y_{k+1} \in Y$ tak, aby vzdálenost prvku $y_{k+1}$ od prvku $y_k$ byla větší nebo rovna jedné. Po přechodu k libovolné posloupnosti dostaneme stále posloupnost, která má prvky vzdálené od sebe více než o jedničku a proto nemůže splňovat B-C podmínku.
\end{proof}

\begin{lemma}[O kompaktnosti identity] \label{3.kompaktnost identity}
Nechť $Y$ je konečněrozměrný Banachův prostor. Pak má $Y$ B-W vlastnost právě tehdy, jestliže je $\map{Id} : Y \mapsto Y$ kompaktní operátor.
\end{lemma}
\begin{proof}
Zřejmé.
\end{proof}

Z předchozích dvou lemmat ovšem dostáváme poměrně překvapivé zjištění: v nekonečnědimenzionálních prostorech není identita kompaktní operátor.

\begin{remark}
V teorii parciálních diferenciálních rovnic se uplatňuje proces \uv{kompaktního vnoření} jednoho prostoru do druhého. V uvedené situaci uvažujeme dva prostory $ X \subset Y$, ovšem opatřené různými normami $\norm{\cdot}_X, \norm{\cdot}_Y$. Zobrazení $\map{Id} : X \mapsto Y$ v takovém případě může být kompaktní. Je to ovšem způsobeno právě růzností norem, které na nekonečnědimenzionálních prostorech nejsou ekvivalentní.

Příkladem takového procesu může být tzv. Rellichova věta:

Nechť $\Omega \subset \R^n$ je otevřená omezená množina s hladkou hranicí. Definujme Sobolevův prostor \begin{align*}
    W^{1,2}(\Omega) := \left \lbrace f: \Omega \mapsto \R : \norm{f}_{W^{1,2}} := \left[ \int_\Omega \left( |f|^2 + |\nabla f|^2 \right) \d x \right]^{1/2} < + \infty \right \rbrace \:.
\end{align*}
Pak platí $W^{1,2}(\Omega) \subset L^2(\Omega)$ a navíc $\map{Id} : W^{1,2}(\Omega) \mapsto L^2(\Omega)$ je kompaktní operátor.

Praktické použití této věty je právě ve vybírání konvergentní podposloupnosti v $L^2$ z omezené posloupnosti v $W^{1,2}$.
\end{remark}

\subsection{Vlastnosti kompaktních operátorů}
V tomto oddíle formulujeme sedm klíčových vlastností, které značně zjednodušují problematiku spektrální analýzy pro kompaktní operátory.

\begin{lemma}[O skládání spojitého a kompaktního operátoru] \label{3.skladani}
Nechť $X$ je Banachův, $\map S \in \lin (X)$ a $\map K \in \comp(X)$. Pak jsou operátory $ \map S \circ \map K$ a $\map K \circ \map S$ kompaktní.
\end{lemma}

\begin{proof}
Zvolme omezenou posloupnost $\sequence{x_n}{n} \subset X$. Pak posloupnost $\sequence{\map S x_n}{n}$ je omezená a díky kompaktnosti $\map K$ můžeme z posloupnosti $\sequence{(\map K \circ \map S) x_n}{n}$ vybrat konvergentní podposloupnost, proto je $\map K \circ \map S$ kompaktní. Dále, z  posloupnosti $\sequence{\map K x_n}{n}$ můžeme vybrat konvergentní $\sequence{\map K x_{n_k}}{k}$ a díky spojitosti operátoru $\map S$ je i podposloupnost $\sequence{(\map S \circ \map K) x_{n_k}}{k}$ konvergentní, proto je $\map S \circ \map K$ kompaktní.

\end{proof}

\begin{lemma}[O příslušnosti nuly do spektra kompaktního operátoru]
Nechť $X$ je nekonečnědimenzionální Banachův prostor, $\map K \in \comp(X)$. Pak číslo $0$ náleží spektru $\Sp(\map K)$.
\end{lemma}

\begin{proof}
Předpokládejme, že $0$ nepatří do $\Sp(\map K)$. Pak podle Věty XY existuje inverzní operátor $ \map K^{-1} \in \lin(X)$. Pak ale díky Lemmatu o skládání kompaktního a spojitého operátoru (Lemma \ref{3.skladani}) dostáváme, že $\map{Id} = \map K \circ \map K^{-1}$ je kompaktní operátor. To je ovšem ve sporu s tvrzením o kompaktnosti identity (Lemma \ref{3.kompaktnost identity}).
\end{proof}

\begin{lemma}[O uzavřenosti obrazu kompaktního operátoru a Fredholmova alternativa]
Nechť $X$ je Banachův prostor, $\map K \in \comp(X)$ je kompaktní operátor a $\lambda \neq 0$. Pak \begin{enumerate}
    \item Obor hodnot operátoru $\mathcal{R}(\map K - \lambda \map{Id})$ je uzavřená množina.
    \item $\map K$ je na (tj. platí $\mathcal{R}(\map K - \lambda \map{Id} )=X$) právě tehdy, když je $\map K - \lambda \map{Id}$ prostý.
\end{enumerate}
\end{lemma}
\begin{remark}
Druhé části předchozího lemmatu se říká "Fredholmova alternativa v nekonečné dimenzi".
\end{remark}

Dosavadní výsledky nám umožňují zjednodušit spektrální tabulku pro kompaktní operátory. DŮSLEDKY, TABULKA, DOPSAT.

\begin{lemma}[O nenulových prvcích spektra kompaktního operátoru]
Nechť $X$ je Banachův prostor, $\map K \in \comp(X)$ je kompaktní operátor a $\lambda \neq 0$ je prvkem $\Sp(\map K)$. Pak \begin{enumerate}
    \item $\lambda$ je vlastní číslo,
    \item $\mathrm{dim} \,(\map K - \lambda \map{Id} ) < + \infty$,
    \item Prostor všech vektorů příslušejících vlastnímu číslu $\lambda$ , $\Ker (\map K - \lambda \map{Id} )$ je uzavřený podprostor $X$.
\end{enumerate}
\end{lemma}

\begin{definition}[Násobnost vlastního čísla]
Číslo $\mathrm{dim} \, \Ker (\map K - \lambda \map{Id} )$ nazýváme násobností vlastního čísla $\lambda$.
\end{definition}

Podle předchozí věty tedy platí, že každé nenulové vlastní číslo má konečnou násobnost - dimenze prostoru vlastních vektorů, které přísluší jednomu nenulovému vlastnímu číslu, je konečná.

\begin{lemma}[O hromadných bodech spektra kompaktního operátoru]
Nechť $X$ je Banachův prostor, $\map K \in \comp(X)$ je kompaktní operátor. Pak je pro každé $\varepsilon>0$ množina \begin{align*}
    \Sp(\map K) \cap \left \lbrace \lambda \in \C : |\lambda| > \varepsilon \right \rbrace
\end{align*}
konečná.
\end{lemma}

\begin{corollary}
Spektrum kompaktního operátoru je vždy nejvýše spočetné. Má-li spektrum hromadný bod, může jím být pouze číslo nula.
\end{corollary}
\pagebreak
