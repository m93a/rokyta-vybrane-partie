\section{Lineární diferenciální operátory}

V této kapitole zavedeme speciální případ lineárních neomezených operátorů, lineární diferenciální operátory, k~nimž budeme generovat báze v polynomiálním tvaru.

\subsection{Výrazy v samoadjungovaném tvaru}

\begin{definition}[Lineární diferenciální výraz n-tého řádu]
Mějme $-\infty \leq a < b \leq \infty$, $y \in \mathcal{C}^n (a, b)$.
Lineárním diferenciálním výrazem n-tého řádu nazveme výraz
\begin{equation}
    l(y) = \sum_{k=0}^{n} p_k (x) y^{(k)},
\end{equation}
kde $\forall k$ je $p_k \in \mathcal{C}(a, b)$ a platí $p_n \not\equiv 0$ na $(a, b)$.
\end{definition}

\begin{remark}
    Ještě se nejedná o lineární diferenciální operátor. K definici lineárního diferenciálního operátoru potřebujeme navíc definiční obor.
\end{remark}

\begin{definition}[Lineární diferenciální operátor]
Pojmem lineární diferenciální operátor $\map L$ n-tého řádu myslíme příslušný lineární
diferenciální výraz $l(y)$ na určitém definičním oboru, tedy
\begin{align*}
    \map L = l \rvert_{\domain(\map L)}
\end{align*}
\end{definition}

Budeme chtít aby $\map L$ byl hustě definovaný v Hilbertově prostoru $H$, tedy $\domain(\map L) \neq H$, $\overline{\domain(\map L)} = H$.
Typicky budeme mít $\domain(\map L) = \left(H \cap \mathcal{C}^n (a,b)\right)$.

Víme, že symetrie je nutnou podmínkou k samoadjungovanosti. Dále budeme hledat další
nutné podmínky samoadjungovanosti, resp. symetrie a budeme pracovat na mnohem menším prostoru $\mathcal{C}^{\infty}_{cpt} (a,b)$, tedy nekonečněkrát diferencovatelné funkce
s kompaktním nosičem.
\begin{remark}
    Díky tomuto přístupu se při provádění per partes nemusíme zabývat okrajovými členy
    jelikož všechny funkce z uvažovaného prostoru $\mathcal{C}^{\infty}_{cpt} (a,b)$ (a jejich derivace) jsou na nějakém okolí krajních bodů nulové. Toto zjednodušení si můžeme dovolit, jelikož hledáme nutné podmínky samoadjungovanosti, resp. symetrie. Pokud operátor nebude splňovat podmínky pro takto \uv{hezké} funkce, nemůže je splňovat pro žádný jiný definičný obor.
\end{remark}

\begin{definition}[Adjungovaný výraz k $l(y)$]
Adjungovaný výraz k výrazu $l(y)$ značíme $l^*(y)$ a definujeme jej jako
\begin{equation}
    l^*(y) = \sum_{k=0}^{n} (-1)^{k} \left(\overline{p_k(y)} ~ y\right)^{(k)}
\end{equation}
\end{definition}

\begin{lemma}
K danému $l$ je $l^*$ jediný lineární diferenciální výraz pro který
\begin{align*}
    \left(l(y), z\right) = \left(y, l^*(z)\right)~ ~ ~ ~ ~ ~ ~ \forall y, z \in \mathcal{C}^{\infty}_{cpt} (a,b).
\end{align*}
\end{lemma}

\begin{proof}
Nejprve dokážeme pomocí per partes rovnost
\begin{align*}
\begin{split}
    \innerprod{l(y)}{z} =& \sum_{k=0}^{n} \int_a^b p_k(x) y^{(k)}(x) \overline{z(x)} =
    \sum_{k=0}^{n} (-1) \int_a^b \left(p_k(x) \overline{z(x)}\right)' y^{(k-1)} = \\
    =& \sum_{k=0}^{n} (-1)^{(k)} \int_a^b \left(p_k(x) \overline{z(x)}\right)^{(k)} y =
    \innerprod{y}{l^*(z)}.
\end{split}
\end{align*}
Poslední rovnost platí díky $p_k(x) \overline{z(x)}= \overline{\overline{p_k(x)} z(x)}$.

Jednoznačnost dokážeme sporem, předpokládejme že existují dva adjungované operátory $l^*$ a $\tilde{l}$ splňující rovnost. Máme
\begin{align*}
    \innerprod{l(y)}{z} = \innerprod{y}{l^*(z)} = \innerprod{y}{\tilde{l}(z)} ~ ~ ~ ~ ~ \forall y, z \in \mathcal{C}^{\infty}_{cpt} (a,b).
\end{align*}
Pak ale nutně z poslední rovnosti, která musí platit pro všechna $y$ z husté podmnožiny v $L^2$, dostáváme
\begin{align*}
    l^*(z) = \tilde{l}(z) ~ ~ ~ ~ ~ ~ \forall z \in \mathcal{C}^{\infty}_{cpt} (a,b),
\end{align*}
pokud si totiž zvolíme pevné $z$, pak ze spojitosti skalárního součinu musí být prvky v druhé složce skalárního součinu stejné. Na husté podmnožině tedy platí $l^*(z) = \tilde{l}(z)$ a z toho plyne
\begin{align*}
    l^* = \tilde{l}.
\end{align*}
\end{proof}
Další podmínkou kterou pro samoadjungovanost potřebujeme je $l = l^*$, což klade podmínku na tvar jednotlivých koeficientů $p_k$
\begin{align*}
    \sum_{k=0}^{n} p_k ~ y^{(k)} = \sum_{k=0}^{n} (-1)^{k} \left(\overline{p_k} ~ y\right)^{(k)} = \sum_{k=0}^{n} (-1)^{k} \sum_{j=0}^{k} {k \choose j} ~ \overline{p_k}^{(k-j)} y^{(j)}.
\end{align*}
Srovnejme koeficienty u $y^{(n)}$
\begin{align*}
    \begin{split}
        &p_n = (-1)^{n} \overline{p_n} \\
        &\text{$n$ sudé} : p_n = \overline{p_n} \implies \text{$p_n$ je reálný} \\
        &\text{$n$ liché} : p_n = - \overline{p_n} \implies p_n + \overline{p_n} = 2 Re (p_n) = 0 \implies \text{$p_n$ je ryze imaginární}
    \end{split}
\end{align*}
a dále postupujeme stejně. Nyní definujeme elementární diferenciální výrazy, pomocí kterých lineární diferenciální výraz $l = l^*$ následně zapíšeme.
\begin{definition}[Elementární diferenciální výraz]
    Elementárním diferenciální výrazem nazveme LDV ve tvaru
    \begin{equation}
        \begin{split}
            E_{2k} &= (-1)^{k} \left(p ~ y^{(k)}\right)^{(k)} \\
            E_{2k-1} &= \frac{i}{2} \left[ \left(p ~ y^{(k-1)}\right)^{(k)} + \left(p ~ y^{(k)}\right)^{(k-1)} \right],
        \end{split}
    \end{equation}
kde $p$ je reálná funkce.
\end{definition}

\begin{theorem}[O kombinaci elementárních diferenciálních výrazů]
    $l(y) = l^{*}(y) ~ \forall y \in \mathcal{C}^{\infty}_{cpt} \iff l(y)$ je konečnou lineární kombinací elementárních diferenciálních výrazů  $E_{2k}$ a $E_{2k-1}$. Více viz \textsc{Čihák}, str. 210.
\end{theorem}

\Priklad{
    Podívejme se, jak elementární diferenciální výraz vYjde pro $k=1$ a $k=2$.
    \begin{equation*}
        \begin{split}
            E_1 &= \frac{i}{2}\left((p~y)' + p~y'\right) = \frac{i}{2} (p'~y + 2 p ~ y') = ip~y' + \frac{i}{2}p' ~ y. ~ \text{ Pro $p=1$ dostáváme }iy'. \\
            E_2 &= -(p~y')' \dots \text{ tzv. diferenciální výraz 2. řádu v samoadjungovaném tvaru}.
        \end{split}
    \end{equation*}
}

\begin{remark}
    Pokud bychom pracovali ve vyšší dimenzi, bude $E_2$ odpovídat základnímu tvaru pro rovnici druhého řádu
    \begin{equation*}
        - \text{div} (p \nabla y).
    \end{equation*}
    Pro $p=1$ pak dostáváme Laplaceův operátor (při omezení na příslušný definiční obor)
    \begin{equation*}
        -\Delta y
    \end{equation*}
\end{remark}

\subsection{Ortogonální báze v $L^{2}_{\rho}$ složené z polynomů}
Budeme uvažovat prostor
\begin{align*}
    H = L^{2}_{\rho} (a, b) \equiv \left\{ f : (a, b) \to \C ; \int_a^b \rho \left|f\right|^2 < \infty, \text{ kde } \rho : (a, b) \to \R \text{ je tzv. \uu{váha} splňující } \rho > 0, \rho \in \mathcal{C}, \rho \in L^1 \right\}.
\end{align*}

\begin{remark}
    Často se uvažuje obecnější váha než $\rho \in \mathcal{C}$.
\end{remark}

Lze ukázat, že $L^{2}_{\rho}(a, b)$ je Hilbertův prostor se skalárním součinem
\begin{align*}
    \innerprod{y}{z}_{2, \rho} \equiv \int_a^b \rho y \overline{z}
\end{align*}
a normou
\begin{align*}
    \norm{y}_{2, \rho} = \int_{a}^{b} \rho \left|y\right|^2.
\end{align*}
\begin{remark}
    Prostor $L^2_{\rho}$ uvažujeme např. proto, že chceme pracovat s polynomy na $\R$. Přitom žádný polynom $P$ není prvkem $L^2(\R)$ ale všechny polynomy jsou prvkem $L^2_{e^{-x^2}}$.
\end{remark}

Uvažujme nyní $\map T : \domain(\map T) \to L^2_{\rho}$, symetrický na $\domain(\map T)$, přitom nechť $\domain(\map T)$ je takové, že $\domain(\map T)\subsetneqq L_\rho^2$, $\overline{\domain(\map T)}=L_\rho^2$ a $\domain(\map T) \subset L^2_{\rho} \cap L^2$.

\begin{definition}[Vlastní číslo a vlastní funkce s vahou $\rho$]
    $\map T$ je hustě definovaný na $\domain(\map T)$. Číslo $\lambda$ nazveme vlastním číslem s vahou $\rho$, pokud $\exists y \neq 0, y \in L^2_{\rho}$ takové, že
    \begin{equation}
        \map T y = \lambda \rho y
    \end{equation}
\end{definition}

Uvažme skalární součin \uu{bez váhy}, tj. i samoadjungovanost \uu{bez váhy}:
\begin{align*}
    &\underset{\rotatebox[origin=c]{90}{=}}{\innerprod{\map Ty}{y}_2 }= \innerprod{\lambda y\rho}{y}_2 = \lambda \innerprod{\rho y}{y}_2 = \lambda \int\limits_a^b\rho |y|^2=\lambda \norm{y}_{2,\rho}^2 \\[-5pt]
    &\innerprod{y}{\map Ty}_2 = \dots = \conj{\lambda}\norm{y}_{2,\rho}^2.
\end{align*}
Pokud je tedy $y\in\domain(\map T)$, je $\lambda \in \R$.

Dále pro $\map Ty_j=\lambda_j\rho y_j$, $j=1,2\dots$, $\lambda_1\neq\lambda_2$, máme
\begin{equation}
    \lambda_1\innerprod{y_1}{y_2}_{2,\rho}= \lambda_1\innerprod{y_1\rho}{y_2}_2 = \innerprod{\map Ty_1}{y_2}_2 = \innerprod{y_1}{\map Ty_2} = \dots = \lambda_2\innerprod{y_1}{y_2}_{2,\rho}.
    \label{eq:ortogonalita}
\end{equation}Jelikož máme různá vlastní čísla $\lambda_1\neq \lambda_2$, dostáváme kolmost v $L_\rho^2$: $\innerprod{y_1}{y_2}_{2,\rho}=0$.

\uu{Závěr:} Uvažujme vl. čísla s vahou a \uv{skalárním součinem bez váhy}. Vlastní vektory pak tvoří ortogonální systém v $L_\rho^2$. Obecně v tomto případě \uu{není k dispozici}
\begin{itemize}
    \item výrok o spočetnosti systému OG funkcí,
    \item výrok o úplnosti báze (musí se dokazovat pro jednotlivé případy zvlášť).
\end{itemize}

Víme však, že generujeme OG množiny, a také máme k dispozici Weistrassovu větu o tom, že polynomy jsou husté v $\comp(K)$ (pro $K$ kompakt). Jelikož $\comp(K)$ je hustá v $L_\rho^2(K)$, zdá se rozumné zabývat systémy OG polynomů v $L_\rho^2$. Pro $L_\rho^2(K)$ pak plyne úplnost OG množiny z Weistrassovy věty. Na nekompaktech se úplnost OG množiny dokazuje obtížněji.

\begin{theorem}[Věta o rekurentním vzorci]
    Mějme prostor $L^2_{\rho}(a,b)$, $-\infty \leq a < b \leq +\infty$, $\rho$ takové, že $\norm{P}_{2,\rho} < \infty ~ \forall$ polynomy $P$. Buď $\varphi_n$ systém reálných OG polynomů v $L^2_{\rho}$, stupeň$(\varphi_n) = n$, $n = 0, 1, 2, 3,\dots$
    Potom $\forall n \in \N ~ \exists A_n, B_n, C_n \in \R$, že
    \begin{equation}
        x \varphi_n = A_n \varphi_{n+1} + C_n \varphi_{n} + B_n \varphi_{n-1}.
    \end{equation}
\end{theorem}

\begin{remark}
    \begin{align*}
        n=0 \implies \varphi_0 = c \neq 0\text{, potom } x \varphi_0 = cx = \frac{c}{a}\underbrace{(ax + b)}_{\;\;\;\varphi_1,\; a\neq0} - \frac{b}{a} \underbrace{c}_{\varphi_0} \implies x \varphi_0 = \frac{c}{a} \varphi_1 - \frac{b}{a} \varphi_0
    \end{align*}
\end{remark}

\begin{proof}
    $n \in \N$ : stupeň$(x \varphi_n) = n+1 \implies ~ \exists \gamma_{n,k} \in \R$, že
   $$
        x \varphi_n = \sum_{k=0}^{n+1} \gamma_{n,k} \varphi_k \quad (\text{obecně polynomy nemusí být OG, postačující podmínkou je st.}(\varphi_n)=n
   $$
    (Toto platí obecně pro jakékoliv polynomy, stupeň$ \varphi_n = n$, nemusí být OG)
    Nyní provedeme skalární součin \uv{s vahou}: $\innerprod{\bullet}{\varphi_j}_{2,\rho} ~ \forall j=0,1,\dots$ a dostaneme
    \begin{align}
        \innerprod{x \varphi_n}{\varphi_j}_{2, \rho} &= \sum_{k=0}^{n+1} \gamma_{n,k}\\ \underbrace{\innerprod{\varphi_k}{\varphi_j}_{2,\rho}}_{\delta_{kj}\norm{\varphi_k}_{2,\rho}^2} &= \begin{cases}
        &\gamma_{n,j} \norm{\varphi_j}^2_{2,\rho} \quad j\leq n+1\\
        &0 \qquad\qquad\quad\;\text{ jinde},
        \end{cases}
    \end{align}
    jelikož $\gamma_{n,j} = 0$ pro $j > n+1$ (suma je rovna nule pro $j>n+1$ protože $\varphi_k \bot \varphi_j$ pro $k \in \{0, \dots, n+1\}$ a $j > n+1$). Alternativně lze definovat $\gamma_{n,k} = 0$ pro $k>n+1$ a sumu psát ve tvaru $\sum_0^\infty$.

Kvůli reálnosti $\varphi_n$ je
$$\innerprod{x\varphi_n}{\varphi_j}{2,\rho} = \innerprod{\varphi_n}{x\varphi_j}{2,\rho} = \innerprod{\varphi_n}{\sum\limits_{p=0}^{j+1}\gamma_{j,p}\varphi_p}{2,\rho} = \sum\limits_{j=0}^{j+1}\gamma_{j,p}\innerprod{\varphi_n}{\varphi_p}_{2,\rho},
$$
což je dle předchozí rovnosti rovno $\gamma_{n,j}\norm{\varphi_j}_{2,\rho}^2$ pro $n>j+1$ a $0$ jinde, tedy $\gamma_{n,j}=0\;\forall j<n-1$.

Celkem $\gamma_{n,j}=0\;\forall j\neq n-1,n,n+1$, z čehož plyne 
$$x\varphi_n = \underbrace{\gamma_{n,n-1}}_{\eqqcolon B_n}\varphi_{n-1} + \underbrace{\gamma_{n,n}}_{\eqqcolon C_n}\varphi_n + \underbrace{\gamma_{n,n+1}}_{\eqqcolon A_n}\varphi_{n+1}.$$

\end{proof}


\begin{remark}
    Navíc lze ukázat, že pro $a=-b$, $\rho$ sudá na intervalu $(a,b)$ je $C_n=0 \;\forall n\in\N$.
\end{remark}
Tento rekurentní vzorec lze např. využít pro vypočet systémů OG polynomů a výpočet jejich norem. Ukažme si to následovně:
$$x\varphi_n=A_n\varphi_{n+1} + C_n\varphi_n + B_n\varphi_{n-1}\quad,n=1,2,3,\dots.$$
Pak lze vypozorovat, že
\begin{itemize}
    \item $A_n\neq0$, jinak je stupeň polynomu vpravo roven $n$.
    \item Projekcí na $\phi_{n+1}$, resp. $\phi_{n-1}$ dostaneme 
        \begin{align*}
            &\innerprod{x\varphi_n}{\varphi_{n+1}}=A_n\norm{\varphi_{n+1}}_{2,\rho}^2\\
            &\underset{\rotatebox[origin=c]{90}{=}}{\innerprod{x\varphi_n}{\varphi_{n-1}}}=A_n\norm{\varphi_{n-1}}_{2,\rho}^2\\[-6pt]
            &\innerprod{x\varphi_{n-1}}{\varphi_n}=A_{n-1}\norm{\varphi_n}_{2,\rho}^2
        \end{align*}
        a z posledních dvou řádků plyne $A_{n-1}\norm{\varphi_n}_{2,\rho}^2=B_n\norm{\varphi_{n-1}}_{2,\rho}^2$ a uvážíme-li $A_n\neq0\;\forall n\in\N\implies B_n\neq0\;\forall n\in 2,3,4,\dots$, dostaneme rekurentní vztah pro normy
        $$\norm{\varphi_{n+1}}_{2,\rho}^2=\frac{B_{n+1}}{A_n}\norm{\varphi_n}_{2,\rho}^2\quad n=1,2,3,\dots$$
        $\norm{\varphi_0}$ a $\norm{\varphi_1}$ samozřejmě musíme znát.
\end{itemize}
Více o neomezených operátorech lze nalézt např. v \textsc{Kreyszig}: Introductory FA with applications. %přidat citace všude KREYSZIG, E. (1978). Introductory functional analysis with applications. New York, N.Y., Wiley. 

\Bonus{
Důkaz poslední poznámky lze provést vyšetřením vlastností polynomů. Máme
$$\varphi_n(-x)=\sum\limits_{k=0}^n\beta_{n,k}\varphi_k(x) \quad\Big/\innerprod{\bullet}{\varphi_j(x)}_{2,\rho} \text{ pro } j=0,\dots,n\text{, jinak je součin } 0$$
\begin{align*}
    \innerprod{\varphi_n(-x)}{\varphi_j(x)}_{2,\rho} &= \beta_{n,j}\norm{\varphi_j}_{2,\rho}^2\\
     \int\limits_{-a}^a\varphi_n(-x)\varphi_j(x)\rho(x)\d x &\overset{ t\coloneqq-x}{\;\;=\;\;} -\int\limits_a^{-a}\varphi_n(t)\varphi_j(-t)\rho(t)\d t \\
     = \innerprod{\varphi_n(x)}{\varphi_j(-x)}_{2,\rho} &=\innerprod{\varphi_n(x)}{\sum\limits_{m=0}^j\beta_{j,m}\varphi_m(x)}=0 \text{ pro }n>j.
\end{align*}
Máme tedy $\varphi_n(-x)=\beta_{n,n}\varphi_n(x)$. Nyní srovnejme koeficienty u $x^n$ v polynomu $\varphi$:
$$a_n(-x)^n=\beta_{n,n}a_nx^n\implies\beta_{n,n}=(-1)^n.$$
Proto $\varphi_n(-x)=(-1)^n\varphi_n(x)\implies (\varphi_n(-x))^2=(\varphi_n(x))^2$, tj. $|\varphi_n|^2$ je \uu{sudá}.

Závěrem máme
\begin{align*}
    x\varphi_n &=A_n\varphi_{n+1}+C_n\varphi_n+B_n\varphi_{n-1}\quad \Big/\innerprod{\bullet}{\varphi_n}_{2,\rho}\\
    \underbrace{\innerprod{x\varphi_n}{\varphi_n}}_{\rotatebox[origin=c]{90}{=}} &= C_n\norm{\varphi_n}_{2,\rho}^2\\[-25pt]
\end{align*}
$$ \overbrace{\int\limits_{-a}^ax|\varphi_n|^2\rho(x)} =0, \quad \text{ neboť } x \text{ jsou lichá a }|\varphi_n|^2_\rho \text{ sudá,}\qquad$$
z čehož plyne $C_n=0$.
}

\subsection{Gaussova redukovaná rovnice a ortogonální systémy polynomů}

Uvažujme tzv. Gaussovu redukovanou rovnici (GRR)
$$\boxed{xy''+(s+1-x)y'- \alpha y=0},\quad x\neq0 ; \;\alpha\in\C ; \;s\neq -1,-2,-3,\dots \text{ (bez spoilerů, důvod brzy poznáme).}$$

\begin{enumerate}

\item Nejprve ukážeme, že tuto rovnici lze psát ve tvaru \uv{vlastní vektor a vlastní číslo s vahou}, tj. ve tvaru 
$$\map Ty=\lambda\rho y\quad \text{pro }\lambda\in\C\text{ s vhodnou vahou }\rho,$$
přičemž $\map Ty$ má tvar diferenciálního výrazu v samoadjungovaném tvaru, tj. $\map Ty=(-py')'$. Tedy
\begin{align*}
    (-py')' &=\lambda\rho y,\quad  p \not \equiv 0\\
    -p'y'-py''-\lambda\rho y&=0\\
    \Aboxed{y''+\frac{p'}{p}y+\lambda\frac{\rho}{p}y&=0}
\end{align*}
Porovnáním poslední rovnosti se zadáním GRR a úpravou pro $x\neq0$ dostaneme
$$y''+\left(\frac{s+1}{x}-1\right)y'-\frac{\alpha}{x}y=0.$$
Jelikož nám stačí nalézt libovolné, netriviální řešení, můžeme psát
\begin{align*}
    \frac{p'}{p}&=\frac{s+1}{x}-1 \\
    \big(\ln|p|\big)' &= (s+1)\big(\ln|x|\big)'-1\\
    |p| &= |x|^{s+1}e^{-x}K\:,\quad K\in\R\\
    p&=x^{s+1}e^{-x}\:,\quad \text{zvolili jsme větev }x>0.
\end{align*}
Pro $x>0$ vyžadujeme $\rho\in\L^1(0,\infty)$, tedy \uu{$s>-1$}. Navíc máme $\lambda=-\alpha,\; \frac{\rho}{p} = \frac{1}{x} \implies \rho=\frac{p}{x},\;\rho = x^se^{-x}$.

Dostali jsme tedy rovnici
\begin{equation}
    \boxed{\big(-\underbrace{x^{s+1}e^{-x}}_{p} y'\big)'=(-\alpha)\underbrace{x^se^{-x}}_{\rho}y\:,}
    \label{eq:SAT}
\end{equation}
jejíž řešení řeší GRR. Je podstatné si uvědomit, že pracujeme na prostoru $L_\rho^2(0,\infty)=L_{x^se^{-x}}^2(0,\infty),\;s>-1$

\item Budeme hledat řešení GRR ve tvaru řady. K tomu však musíme učinit násedující úvahy.
    \begin{itemize}
        \item  Pro $x=0$ má GRR degenerovaná řešení, je potřeba ji vyšetřovat separátně na $(-\infty,0)$ a na $(0,\infty)$.
        \item Můžeme předpokládat, že tato dvě separátní řešení bude možné \uv{slepit} v bodě $x=0$ tak, že vznikne řešení na nějakém $(-K,K)\subset \R$. Pokud hledáme řešení GRR ve třídě takovýchto \uv{slepitelných} řešení, lze je hledat i ve tvaru Taylorovy řady se středem v nule. S tím rizikem, že řešení v tomto tvaru nenajdeme, což by nás dovedlo k závěru, že úloha žádná \uv{slepitelná} řešení ve tvaru řady nemá.
    \end{itemize}
Za této podmínky položíme $y\eqqcolon \sum\limits_{n=0}^\infty c_nx^n$ a dosazením do GRR dostaneme
\begin{align*}
    \sum\limits_{n=2}^\infty c_nn(n-1)x^{n-2}x + (s+1)\sum\limits_{n=1}^\infty c_n nx^{n-1}-\sum\limits_{n=1}^\infty c_nnx^n-\alpha\sum\limits_{n=0}^\infty c_nx^n&=0\\
    \sum\limits_{n=1}^\infty c_{n+1}(n+1)nx^x + \sum\limits_{n=0}^\infty (s+1)c_{n+1} (n+1)x^n-\sum\limits_{n=1}^\infty c_nnx^n-\sum\limits_{n=0}^\infty c_n\alpha x^n&=0
\end{align*}
Nyní srovnejme koeficienty:
\begin{align*}
    x^0:\quad (s+1)c_1=c_0\alpha \implies c_1=c_0\frac{\alpha}{s+1} \quad &\text{(pro }s\neq-1,\dots\text{)}\\
    \text{pro }n>1\text{ máme } x^n:\quad c_{n+1}\left[(n+1)n+(s+1)(n+1) \right] = c_n&(n+\alpha) \\
    c_{n+1}=c_n&\frac{n+\alpha}{(n+1)(s+n+1)} \quad(s\neq-2,-3,\dots)\:,
\end{align*}
přičemž získaný rekurentní vztah je platný i pro $n=0$. Protože každý násobek řešení GRR je zase jejím řešením, lze BÚNO volit základní řešení pro $c_0=1$. Dostáváme, že koeficienty řady, která definuje řešení, by musely mít tvar
\begin{align*}
    c_0&=1\\
    c_{n+1}&=\frac{n+\alpha}{n+s+1}\frac{c_n}{n+1}\:,\quad \text{pro } n=0,1,2,\dots; s\neq-1,-2,\dots \:.
\end{align*}

Ještě však musíme ukázat, že řada s takto danými koeficienty $c_0,\;c_1$ někde konverguje. Protože řady s koeficienty tohoto typu tvoří jednu velmi důležitou třídu řad, věnujeme jim následující intermezzo.


\end{enumerate}


\newpage
\hrulefill\\
\uu{Intermezzo: Hypergeometrické řady}

\begin{definition}[Hypergeometrická řada]
    Hypergeometrickou řadou nazveme mocninnou řadu tvaru
    $$\sum\limits_{n=0}^\infty c_nx^n,$$
    pro koeficienty splňující
    \begin{itemize}
        \item $\exists$ polynomy $P,Q$ s koeficienty u nejvyšší mocniny rovnými 1, $\st P=p\geq0$, $\st Q=q\geq0$, $Q$ nemá kořeny mezi $\N\cup\{0\}$,
        \item $\frac{c_{n+1}}{c_n}=\frac{P(n)}{Q(n)}\frac{1}{n+1}\:,\quad n=0,1,2,\dots\:;\; c_0=1$
    \end{itemize}
\end{definition}

\begin{remark}
    Pro $P(n)=Q(n)(n+1)$ máme $\frac{c_{n+1}}{c_n}=1$, $\left|\frac{c_{n+1}x^{n+1}}{c_nx^n}\right|=|x|$, tedy geometrickou řadu s kvocientem $x$. Člen $\frac{1}{n+1}$ je zde jen z historických důvodů.
\end{remark}
Rozložme nyní $P$ a $W$ na kořenové činitele v $\C$ a dostaneme
\begin{equation}
    \frac{c_{n+1}}{c_n} = \frac{(a_1+n)(a_2+n)\dots(a_p+n)}{(b_1+n)(b_2+n)\dots(b_q+n)}\frac{1}{n+1}\:.    
    \label{eq:koeficienty}
\end{equation}
Pro tuto situaci se zavádí tzv. \uv{Identický zápis hypergeometrické řady}
$$\sum\limits_{n=9}^\infty c_nxx^n = {}_pF_q[a_1,\dots,a_p;b_1,\dots,b_q](x).$$
Z rovnice \ref{eq:koeficienty} ihned vidíme:
\begin{itemize}
    \item $p<q+1\implies \left|\frac{c_{n+1}}{c_n}\right| \rightarrow 0\implies R=+\infty\implies \sum c_nx^n$ definuje holomorfní ($\mathcal{C}^\infty$) fci na celém $\C$.
    \item $p=q+1\implies \left|\frac{c_{n+1}}{c_n}\right| \rightarrow 1\implies R=1\implies \sum c_nx^n$ definuje holomorfní ($\mathcal{C}^\infty$) funkci na $\mathcal{U}^2(0)$.
    \item $p>q+1\implies \left|\frac{c_{n+1}}{c_n}\right| \rightarrow \infty\implies R=0$, což nedefinuje žádnou diferencovatelnou funkci.
\end{itemize}
\begin{definition}[Pochhammerův symbol]
    Pro $a\in\C$ definujme 
    \begin{align*}
        (a)_0 \coloneqq 1\:,\quad   (a)_n\coloneqq \underbrace{a(a+1)\dots(a+n-1)}_{n\in\N \text{ členů}}
    \end{align*}
\end{definition}
Pochhammerovu symbolu se někdy též říká \uv{Rising factorial} a užívá se i značení $\langle a\rangle_n$. Budeme ho číst \uv{$a$ Pochhammer $n$}, případně \uv{$a$ dole $n$}. Nakonec si ještě všimněte, že např. $(1)_n=n!$, nebo $(a)_n=\frac{\Gamma(a+n)}{\Gamma(a)}$.

S tímto značením přepíšeme rovnici \ref{eq:koeficienty}
\begin{align*}
    c_n &= \frac{(a_1+n-1)(a_2+n-1)\dots(a_p+n-1)}{(b_1+n-1)(b_2+n-1)\dots(b_q+n-1)}\frac{1}{n}c_{n-1} \\
    &=  \frac{[(a_1+n-1)(a_2+n-2)]\dots[(a_p+n-1)(a_p+n-2)]}{\underbrace{[(b_1+n-1)(b_2+n-2)]}_{\text{dalšími kroky jde k }(b_1+n-1)\dots(b_1) = (b_1)n}\dots[(b_q+n-1)(b_q+n-2)]}\underbrace{\frac{1}{n}\frac{1}{n-1}}_{\text{jde k }\frac{1}{n!}} c_{n-2} = \\
    &\frac{(a_1)_n\dots(a_p)_n}{(b_1)_n\dots(b_q)_n}\frac{1}{n!}\underbrace{c_0}_{=1} = \frac{\prod\limits_{j=1}^p(a_j)_n}{\prod\limits_{k=1}^q(b_k)_n}\frac{1}{n!}\:.
\end{align*}
Tím jsme se dostali k \uu{explicitnímu vyjádření hypergeometrické řady}
$$\boxed{{}_pF_q[a_1,\dots,a_p;b_1,\dots,b_q](x) = \sum\limits_{n=0}^\infty \frac{\prod\limits_{j=1}^p(a_j)_n}{\prod\limits_{k=1}^q(b_k)_n}\frac{x^n}{n!}}$$


Důvod přidání onoho historického členu $\frac{1}{n+1}$ vyniká v hypergeometrické řadě ${}_0F_0[;](x)$. Ta je dle předchozího závěru rovna
$${}_0F_0[;](x)=\sum\limits_{n=0}^\infty\frac{x^n}{n!}=e^x$$

\Priklad{
\begin{itemize}
    
\item  Ukažte:
$${}_0F_0\left[;\frac{1}{2}\right]\left(- \frac{x^2}{4}\right)=\cos x.$$
Řada vlevo má $p=0$, $q=1\implies p<q+1\implies$ řada definuje hladkou (a holomorfní) funkci v $\C$.


\textcolor{gray}{
Řešení:
\begin{align*}
    {}_0F_1\left[;\frac{1}{2}\right]\left(-\frac{x^2}{4}\right) &=\sum\limits_{n=0}^\infty\frac{1}{(1/2)_n}\frac{1}{n!}\left(-\frac{x^2}{4}\right)^n\\
    &= \sum\limits_{n=0}^\infty(-1)^nx^{2n}   \underbrace{\frac{1}{n!4^n\frac{1}{2}\left(\frac{1}{2}+1\right) \dots \left(\frac{1}{2}+n-1\right)}}_{\frac{2^n}{n!4^n(1\cdot 3\dots(2n-1))} = \frac{1}{(2n)!}} = \sum\limits_{n=0}^\infty(-1)^n\frac{x^{2n}}{(2n)!}
\end{align*}
}

\item Ukažte 
$$\frac{2x}{\sqrt{\pi}}{}_1F_1\left[\frac{1}{2};\frac{3}{2}\right]\left(-x^2\right) =\frac{1}{\sqrt{\pi}} \int\limits_0^xe^{-t^2}\d t = \erf (x)\:,$$
přičemž řada vlevo má smysl pro $\forall x\in\C$, avšak pravá má smysl pouze pro $x\in\R$. Řadu nalevo lze tedy chápat, jako rozšíření $\erf (x)$ do komplexních čísel.

\end{itemize}


}

konec intermezza. Vraťme se k GRR.


\hrulefill


\begin{itemize}
    \item[3.] Řešením GRR je řada $\sum_{n=0}^\infty c_nx^n$, kde $$c_0=1\:,\quad c_{n+1} = \frac{n+\alpha}{n+s+1}\frac{1}{n+1}c_n\:,\quad n=0,1,2,\dots\:.$$
    Jde tedy o hypergeometrickou řadu pro $p=1$, $q=1$, tj. $p<q+1$ a máme
    $${}_1F_1\left[\alpha;\beta+1\right](x) = \sum\limits_{k=0}^\infty\frac{(\alpha)_k}{(s+1)_k}\frac{x^k}{k!}\in\mathcal{C}^\infty(\R)\:,\quad \alpha\in\C;\;s\in\C\setminus\{-1,-2,-3,\dots\}$$
    
${}_1F_1[\alpha,s+1](x)$ je polynomem, právě když řada vpravo obsahuje konečný počet členů, což nastane právě když $\exists n\in\N,\; (\alpha)_k=0\;\forall k>n$.
Navíc jelikož $(\alpha)_k = \alpha(\alpha+1)\dots(\alpha+k-1)$, dostaneme pro $\alpha=-n$ ekvivalenci
$$(\alpha)_k=0\;\Leftrightarrow\;k>n \text{ pro } n\in\N\cup\{0\}\:.$$
\begin{definition}[Laguerrův polynom]
    Laguerrův polynom řádu $s$ a stupně $n$ je polynom, definovaný pro $x,s\in\R$, $s>-1$, jako
    $$\boxed{L^s_n(x)\eqqcolon \frac{(s+1)_n}{n!}{}_1F_1[-n;s+1](x) = \frac{(s+1)_n}{n!}\sum\limits_{k=0}^n\frac{(-n)_k}{(s+1)_k}\frac{x^k}{k!}\:.}$$
\end{definition}

\begin{remark}\qquad\qquad\qquad\qquad

    \begin{itemize}
        \item $L_n^s(x)$ řeší GRR $\forall n\in\N\cup\{0\}$, pokud v ní položíme $\alpha=-n$.
        \item S odvoláním na tvar rovnice \ref{eq:SAT} provedeme následující restrikce:
        \begin{itemize}[\textbullet]
            \item Uvažujeme $x>0$, tj. $x\in (0,\infty)$
            \item Uvažujeme $s\in \R$, $s>-1$, $\rho(x)=x^se^{-x}$. Pak $\rho>0$ na $(0,\infty)$, $\rho\in\mathcal{C}(0,\infty)\cap L^1(0,\infty)$, což ukazuje na oprávněnost volby $\rho$, jakožto váhy.
            \item Uvažujeme tedy Hilbertův prostor $L^2_{x^se^{-x}}(0,\infty)$.
            \item $\alpha=-n$, $n\in\N\cup\{0\}$.
        \end{itemize}
    \end{itemize}
    Potom budeme GDD psát v samoadjungovaném tvaru (rovnice \ref{eq:SAT})
    $$\map Ty = \eta \rho y\:,$$
    kde $\map Ty=-(py')'$, $p(x)=x^{s+1}e^{-x}$.
    
    \boxed{\begin{aligned}
    &\text{Potom } n=0,1,2,\dots \text{ jsou vlastní čísla } \map T \text{ s vahou } \rho \text{(na }L_\rho^2(0,\infty)\text{) a jim odpovídající vlastní funkce jsou}\\
    &\text{ Laguerrovy polynomy }L_n^s.
    \end{aligned}}
    
    \item Podle výpočtu provedeného dříve (viz úvahy za rovnicí \ref{eq:ortogonalita}) tvoří Laguerrovy polynomy (pro pevní $s>-1$ a pro $n\in\N\cup\{0\})$ OG systém polynomů v $L^2_{x^se^{-x}}(0,\infty$. Existuje tedy rekurentní vzorec pro jejich vygenerování (odvodíme dále).
    \item Laguerrovy polynomy jsou navíc \uu{úplným} systémem, tj. každou funkci z $L^2_{x^se^{-x}}(0,\infty)$ lze napsat ve tvaru $\sum_{n=0}^\infty c_nL_n^s(x)$. Důkaz je nad rámec těchto skript a lze nalézt např. v \textsc{Čihák a kol}: MA pro fyziky V. (Věta 31, str 196).
\end{remark}

\end{itemize}

\subsection{Některé důležité vlastnosti Laguerrových polynomů}

\subsubsection{Tzv. explicitní vyjádření}
Platí 
\begin{equation}
    \boxed{L^s_n(x) = \frac{1}{n!} x^{-s} e^x \left(x^{s+n} e^{-x}\right)^{(x)}}
    \label{eq:explicit}
\end{equation}
Z toho plyne, že
\begin{align*}
L^s_0(x)&=x^{-s}e^xx^se^x=1\\
L^s_1(x) &= x^{-s}e^x(x^{s+1}e^{-x})' = x^{-s}e^x(s+1)x^se^{-x}+x^{-s}e^xx^{s+1}(-e^{-x}) = (s+1)-x \text{ atd.}
\end{align*}

Dalším využití explicitního vyjádření \ref{eq:explicit} se nabízí při výpočtech integrálů typu
$\int\limits_0^\infty L_n^s(x)f(x)\d x\:,$
protože umožňuje použití metody per-partes.
\begin{proof}
    Důkaz vyjádření \ref{eq:explicit} začneme úpravou GRR
    \begin{align*}
        xy''+(s+1-x)y'-\alpha y&=0 \\
        \left(s^{s+1}e^{-x}y'\right)' &=\alpha x^se^{-x}y\:,  \text{ označme rovnost jako }\GRR(y,s+1,\alpha)\:.
    \end{align*}
    Derivací dostaneme
    \begin{align*}
        xy'''+y''+(s+1-x)y''-y'-\alpha y'&=0\\
        xy'''+(s+2-x)y''-(\alpha+1)y'&=0\:, \text{ rovnost označme: } \GRR(y',s+2,\alpha+1)\:.
    \end{align*}
\end{proof}
$n-1$ derivacemi tak dostaneme
$$\GRR(y^{(n-1)},s+n,\alpha+n-1) \equiv \underbrace{\left(x^{s+n}e^{-x}y^{(n)}\right)'}_{\eqqcolon V_n} = (\alpha+n-1)\underbrace{x^{s+n-1}e^{-x}y^{(n-1)}}_{V_{n-1}}\:,$$
tedy $V_n'=(\alpha+m-1)V_{n-1}$ a dalším derivováním máme
$$V_n''=(\alpha+m-1)V_{n-1}' = (\alpha+n-1)(\alpha+n-2)V_{n-2}.$$
Postupně pak máme
$$V_n^{(n)}=(\alpha)_nV_0 = (\alpha)_nx^se^{-x}y\:,$$
z čehož máme
$$\left(x^{s+n}e^{-x}y^{(n)}\right)^{(n)} = (\alpha)_nx^se^{-x}y$$
a dosazením $\alpha\coloneqq -n$ dostaneme
\begin{equation}
    y = \frac{1}{(-n)_n}x^{-s}e^x\left(x^{s+n}e^{-x}y^{(n)}\right)^{(n)}\:.
    \label{eq:B}
\end{equation}
Pokud je $\alpha=-n$ řešením $L^s_n$, což je polynom stupně $n$. Jeho $n-$tá derivace je tedy konstanta, $(L_n^s)^{(n)}=a_n n!$, kde $a_n$ značí koeficient u $x^n$. Polynom lze zapsat řadou
$$L_n^s(x) = \frac{(s+1)_n}{n!}\sum\limits_{k=0}^n\frac{(-n)_l}{(s+1)_k}\frac{x^k}{k!}\:,$$
tedy $a_n = \frac{(s+1)_n}{n!}\frac{(-n)_n}{(s+1)_n}\frac{1}{n!}$, proto $(L_n^s)^{(n)} = \frac{(-n)_n}{n!}$.

Dosazením do rovnice \ref{eq:B} máme
\begin{align*}
    L_n^s(x)&=\frac{1}{(-n)_n}x^{-s}e^x\left(x^{s+n}e^{-x}\frac{(-n)_n}{n!}\right)^{(n)}
\end{align*}
\begin{equation}
    L^s_n(x) =\frac{1}{n!}x^{-s}e^x\left(x^{s+n}e^{-x}\right)^{(n)}
    \label{eq:C_dukazend}
\end{equation}


\subsubsection{Rekurentní vzorec pro $L^s_n(x)$}
Vyjdeme z rovnice \ref{eq:C_dukazend}
$$L_n^s(x)=\frac{1}{n!}x^{-s}e^x\underbrace{\left(x^{s+n}e^{-x}\right)^{(n)}}_{\eqqcolon E_n}\:.$$
Pak přímým derivováním dostaneme
\begin{equation}
    E_{n+1} = \left( \left(x^{s+n+1}e^{-x}\right)'\right)^{(n)}= (s+n+1)\underbrace{\left(x^{s+n}e^{-x}\right)^{(n)}}_{E_n} - \underbrace{\left(x^{s+n+1}e^{-x}\right)^{(n)}}_{\eqqcolon I_n}\:.
    \label{eq:D_Enp1}
\end{equation}
Našim cílem je nyní vyjádřit $I_n$ pomocí $E_n$.
\begin{align*}
    I_n &= \left(x x^{x+n}e^{-x}\right)^{(n)} = \sum\limits_{k=0}^n\binom{n}{k} x^{(k)}(x^{s+n}e^{-x})^{(n-k)} = \big/\text{je nenulové jen pro }k\in\{0,1\}\big/ \\
    &= x\left(x^{s+n}e^{-x}\right)^{(n)} + n\left(x^{s+n}e^{-x}\right)^{(n-1)} = xE_n+ n \underbrace{(x^{s+n}e^{-x})^{(n-1)}}_{I_{n-1}}\:,
\end{align*}
tj. $I_n = xE_n+n I_{n-1}$.

Vezmeme-li navíc \ref{eq:D_Enp1} pro $n-1$, tedy $E_n=(s+n)E_{n-1}-I_{n-1}$, dostaneme $$I_n = xE_n + n(s+n)E_{n-1}-n E_n\:.$$ 
Tuto rovnici dosadíme do \ref{eq:D_Enp1}, čímž dostaneme
\begin{align*}
    E_{n+1} &= (s+n+1)E_n - xE_n-n(x+n)E_{n-1} + nE_n\\
    x E_n &= (s+2n+1)E_n-E_{n+1} - n(s+n)E_{n-1} \quad \Big/\cdot \frac{1}{n!}x^{-s}e^x\\
    \Aboxed{ xL_n^s(x) &= (s+2n+1)L_n^s(x) - (n+1)L^s_{n+1}(x)-(s+n)L_{n-1}(x)\:,}
\end{align*}
což je hledaný rekurentní vzorec. Ze znalosti prvních dvou členů ($L_0^s=1$, $L_1^s=(s+1)-x$) lze vygenerovat všechna $K_n^s$.


\subsubsection{Normy}
Z dřívějška víme
$$\norm{\varphi_{n+1}}_{2,\rho}^2 = \frac{B_{n+1}}{A_n}\norm{\varphi_n}_{2,\rho}^2\:, \quad n=1,2,\dots\:,$$
pokud  $$y\varphi_n = A_n\varphi_{n+1}+C_n\varphi_n+B_n\varphi_{n-1}\:.$$
Zde máme $A_n=-(n+1)$, $B_n=-(s+n)$, tedy
$$\norm{L_{n+1}}_{2,\rho}^2 = \frac{s+n+1}{n+1}\norm{L_n^0}_{2,\rho}^2\:, \quad
 n=1,2,3,\dots\:.$$
 Máme $\norm{L_0^s}_{2,\rho}^2 = \int\limits_0^\infty 1\cdot x^se^{-x}=\Gamma(s+1)$.
 \begin{align*}
     \norm{L_1^s}_{2,\rho}^2 = \int\limits_0^\infty\left((s+1)-x\right)^2x^se^{-x} &= (s+1)^2\Gamma(s+1)-2(s+1)\Gamma(s+2)+\Gamma(s+3)\\
     &=(s+1)\Gamma(s+2)-2(s+1)\Gamma(s+2)+\Gamma(s+3)\\
     &=\Gamma(s+3)-(s+1)\Gamma(s+2)\\
     &=(s+2)\Gamma(s+2)-(s+1)\Gamma(s+2) = \Gamma(s+2)
 \end{align*}
a rekurencí
\begin{align*}
    \norm{L_n^s}_{2,\rho}^2=\frac{s+n}{n}\frac{s+n-1}{n-1}\dots \frac{s+2}{2}\underbrace{\norm{L_1^s}_{2,\rho}^2}_{\Gamma(s+2)}=\frac{1}{n!}\Gamma(s+n+1) \text{ , což platí i pro }n\in\{0,1\}\\
    \Aboxed{\norm{L_n^s}_{2,\rho}^2=\frac{1}{n!}\Gamma(s+n+1)\quad \forall n=0,1,2\dots}
\end{align*}


\subsubsection{Tzv. vytvořující funkce}
\begin{definition}[Vytvořující funkce]
    Vytvořující funkcí pro daný systém $\{\varphi_n\}_{n=0}^\infty,\; \varphi=\varphi_n(t)$, nazvu takovou funkci $F=F(x,t)$, která je analytická v okolí $t=0$ (pro všechna pevná $x$) a jejíž rozvoj do Taylorovy řady podle $t$ v $t\in\mathcal{U}(0)$ generuje koeficienty $\varphi_n(x)$. Tedy
    $$F(x,t)=\sum\limits_{n=0}^\infty\varphi_n(x)t^n\:.$$
\end{definition}

Zde tedy hledáme takovou $F$, pro kterou $F(x,t)=\sum\limits_{n=0}^\infty L_n^s(x)t^n$. 

Budeme postupovat tak, že rozvineme vhodnou funkci $f\in L_\rho^2(0,\infty)$ s parametrem $t$ do řady v Laguerových polynomech. Tím dostaneme řadu typu $\sum_{n=0}^\infty c_n(t)L_n^s(x)$ a budeme směřovat k tomu, aby $c_n\approx t^n$.

Teorie říká, že pokud $f\in L^2_{x^se^{-x}}(0,\infty)$ a $L_n^s(x)$ je úplný v $K^2_{x^se^{-x}}(0,\infty)$, tak z obecné teorie Fourierových řad plyne, že $\exists c_n\in\C$ tvaru
$$c_n=\frac{1}{\norm{L_n^s}_{2,\rho}^2} \innerprod{f}{L_n^s}_{2,\rho}\:,\text{ že } f=\sum c_nL_n^s\:,$$ 
přičemž rovnost u sumy je myšlena v $L_{x^se^{-x}}(0,\infty)$.

Budeme rozvíjet funkci $e^{-ax}$ (poté budeme hledat $a=a(t)$).

\uu{První krok} spočívá v nalezení podmínky pro $a\in\R$, kdy je $e^{-ax}\in L^2_{x^se^{-x}}(0,\infty)$, tj kdy 
\begin{align*}
    \int\limits_0^\infty(e^{-ax})^2x^se^{-x}\d x &<\infty \\
    \int\limits_0^\infty x^se^{-(2a+1)x}\d x<\infty \text{ pro } s>-1 \text{, pokud }2a+1>0 \Leftrightarrow a>-\frac{1}{2}\:.
\end{align*}
Pro toto $a$ spočteme
\begin{align*}
    c_n&=\frac{1}{\norm{L_n^s}_{2,\rho}^2}\int\limits_0^\infty e^{-ax}x^se^{-x}L_n^s(x)\d x = \big/\text{explicitní vyjádření }L_n^s\big/ \\
    &=\frac{n!}{\Gamma(s+n+1)}\int\limits_0^\infty e^{-ax}x^se^{-x}\left(\frac{1}{n!}x^{-s}e^x(x^{s+n}e^{-x})^{(n)}\right)\d x\\
    &=\frac{1}{\Gamma(s+n+1)}\int\limits_0^\infty (e^{-ax}x^{s+n}e^{-x})^{(n)}\d x = \big/n\times\text{ per partes, hraniční členy jsou nulové}\big/\\
    &=\frac{a^n}{\Gamma(s+n+1}\int\limits_0^\infty \underbrace{e^{-ax}x^{s+n}e^{-x}}_{(a+1)x = y}\d x \\
    &=\frac{a^n}{\Gamma(s+n+1)}\int\limits_0^\infty e^{-y}\left(\frac{y}{a+1}\right)^{s+n}\frac{1}{a+1}\d y \\
    &=\frac{a^n}{\Gamma(s+n+1}\frac{1}{(a+1)^{s+n+1}}\Gamma(s+n+1)=\frac{1}{(a+1)^{s+1}}\left(\frac{a}{a+1}\right)^n\:.
\end{align*}
Odtud dostáváme
\begin{equation}
    \boxed{e^{-ax}\overset{\text{s.v.}}{=}\frac{1}{(a+1)^{s+1}}\sum\limits_{n=0}^\infty \left(\frac{a}{a+1}\right)^n L_n^s(x)\:,\quad a>-\frac{1}{2}\:.
}
\label{eq:rada_e^-ax}
\end{equation}

\begin{remark}
\begin{itemize}
    \item Obecně platí rovnost ve smyslu prostoru, ve kterém byla odvozena, tj v $L^2_{x^se^{-x}}(0,\infty)$, neboli skorovšude (s.v.). Pokud jsou však na obou stranách spojité funkce (např řada vpravo konverguje alespoň lokálně stejnoměrně v $\R$, platí rovnost ve všech $x\in\R$.
    \item Dosazením $a=1$ do rovnice \ref{eq:rada_e^-ax} vypadnou všechny členy pro $n\geq1$ a dostaneme $1=L_0^s(x)$.
    \item Pro $a=1$ dá rovnice \ref{eq:rada_e^-ax}
    $$e^{-x}=\frac{1}{2^{s+1}}\sum\limits_{n=0}^\infty\frac{L_n^s(x)}{2^n}\:,$$
    speciálně pro $s=0$ máme 
    $$e^{-x}=\sum\limits_{n=0}^\infty \frac{L_n^0(x)}{2^{n+1}}\:.$$
\end{itemize}
\end{remark}



\uu{Druhý krok} je sestavení vytvořující funkce.

Položme $t=\frac{a}{a+1}$ ($\Leftrightarrow a=\frac{t}{1-t}$) v rovnici \ref{eq:rada_e^-ax}. Derivace tohoto výrazu $\frac{\d t}{\d a}=\frac{1}{(a+1)^2}>0$ je prostá. Platí $a>-\frac{1}{2}$ právě pro $t\in(-1,1)$. Úpravou rovnice \ref{eq:rada_e^-ax} pak máme 
\begin{align}
(a+1)^{s+1}e^{-ax}&=\sum\limits_{n=0}^\infty L_n^s(x)\left(\frac{a}{a+1}\right)^n \quad \Big/n\rightarrow t \\
\underbrace{\frac{1}{(1-t)^{s+1}}e^{-\frac{tx}{1-t}}}_{\text{vytvoř. fce pro Laguerrovy p.}}&=\sum\limits_{n=0}^\infty L_n^s(x) t^n\:,\quad t\in(-1,1)\:,
\end{align}
V tabulce \uv{Ortogonální systémy polynomů} v dodatku jsou uvedeny Laguerrovy, hermitovy, Legendreovy, Čebyševovy a Gegenbauerovy systémy polynomů. Vždy uvádíme generující funkci, vyjádření řadou, explicitní tvar, rekurentní vztah a velikosti norem, vytvořující funkci a zejména prostor, ve kterém tvoří bázi.



