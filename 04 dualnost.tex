%
%Několik poznámek ke značení:
%Pro přehlednost jsem se snažil psát u symbolu normy a skalárního součinu, v jakém prostoru se vlastně norma tvoří. Při učení několika dlouhých operátorových nerovností mi tato drobnost pomohla.
%Rozhodl jsem se rovněž označit nějakým způsobem lineární operátory a zobrazení. Symbol \lin používám pro označení prostoru spojitých lineárních zobrazení a samotná je označuji \map.
%V kapitole duální operátor se ovšem těžko rozlišuje mezi prvky prostoru a jeho duálu. Značení ponechávám, ale lze ho kdykoli vylepšit nebo tak něco.
%Při psaní se držím souboru s originálními poznámkami, ale také svých vlastních poznámek. Některé rozčlenění tyou "Poznámka vs Příklad vs Hlavní text" jsem trochu přizpůsobil k svému obrazu, snad to není na škodu.





\section{Duálnost}

\subsection{Duál a dualita}

\begin{definition}[Duál]
Buď $X$ Banachův prostor. Prostor $X' \coloneqq \lin(X,\C)$ \hspace{0.5em} (resp. $\lin(X, \R)$) nazýváme topologickým duálem k $X$.
\end{definition}

\begin{remark} \ph{.}
\begin{itemize}
    \item Prostor $X'$ je tedy tvořen všemi spojitými lineárními funkcionály, spojitost uvažujeme ve smyslu \begin{align*}
        x_n \xrightarrow{X} x \implies \map T x_n \to \map T x \quad \text{pro všechna } \map T \in X' \:.
    \end{align*}
    \item Víme, že jsou-li $X,Y$ normované prostory a $Y$ je Banachův, pak i $\lin (X,Y)$ je Banachův. Proto je $X'$ automaticky Banachovým prostorem.
    \item Je-li $\map T \in X'$, pak jeho norma je přirozeně $\norm{\map T}_{X'} := \underset{\norm{x}_X \leq 1}{\sup} |\map Tx|$.
    \item Topologický duál není totéž, co \uu{vektorový duál} (pouze lineární zobrazení $X \to \C$ ($\R$), nevyžaduje spojitost). Prvků vektorového duálu je víc (o ony „nespojité“). V konečné dimenzi pro $X$ Banachův je vektorový duál vždy topologickým duálem.
\end{itemize}
\end{remark}


\begin{definition}[Dualita]
Nechť $X$ je Banachův a $X'$ jeho duál. Zobrazení $\map S : X \times X' \mapsto \C$ nazveme dualitou, jestliže splňuje vlastnosti: \begin{enumerate}
    \item sekvilinearita: $\map S(\alpha x + \beta y, z) = \alpha \map S(x,z) + \beta \map S(y,z), \;\; \map S(z,\alpha x + \beta y) = \conj{\alpha} \map S(z,x) + \conj{\beta} \map S(z,y)$.
    
    \item spojitost: $\map S(x_n,y_n) \rightarrow \map S(x,y) \quad \text{kdykoliv } (x_n,y_n) \rightarrow (x,y) \text{ v } X \times X'$.
\end{enumerate}
Někdy píšeme $\map S(x, \map T) = \duality{x}{\map T}$. Obecně je často $\duality{\cdot}{\cdot}$ symbolem duality.
\end{definition}

\begin{remark}
Jsme-li v nějakém smyslu schopni ztotožnit prostory $X$ a $X'$ (uvidíme v následujícím), pak roli duality hraje skalární součin. Ztotožněním prostorů myslíme následující: píšeme $X \simeq Y$, jestliže existuje zobrazení $\map D: X \mapsto Y$, které je izometrické a izomorfní, tj. zachovává normu a je bijektivní.
\end{remark}

\begin{example}

Uvažujme prostor vektorů $\R^n$. Víme, že každá lineární forma $\map T \in {\R^n}'$ je reprezentovatelná lineární kombinací (násobení transponovaným vektorem): $\map T \left[ (x_1, \cdots, x_n) \right] = \sum_{j=1}^n \alpha_j x_j$. Transponování je izometrické a izomorfní zobrazení, lze tedy psát $\R^n \simeq {\R^n}'$. Dualitou na takovém prostoru je například \textit{skalární součin} na $\R^n$. Později uvidíme, že podobně lze uvažovat i v jakémkoliv Hilbertově prostoru – v tomto smyslu je dualita zobecněním skalárního součinu.
\end{example}

\begin{theorem}[O ztotožnění sdružených Lebesgueových prostorů]
Buď $\Omega \subseteq \R^n $ otevřená souvislá množina, $\map T \in L^q(\Omega)'$. Nechť pro $p \in (1, \infty)$ platí $\frac{1}{p} + \frac{1}{q} =1$ ($p$ je tzv. sdružený exponent ke $q$). Pak existuje právě jeden prvek $g \in L^p(\Omega)$ takový,~že:
$$
\map T(f) = \int_\Omega f \conj{g} \d x \quad \forall f \in L^p(\Omega)
\quad \text{a zároveň} \quad
\norm{\map T}_{L^p(\Omega)'} = \norm{g}_{L^q(\Omega)} \:.
$$

\end{theorem}

\begin{proof}
Dokázat, že takové zobrazení existuje, nedá příliš práce. Dokázat jeho jednoznačnost je však velmi pracná záležitost.
\end{proof}

Předchozí věta ukazuje, že platí $L^p(\Omega)' \simeq L^q(\Omega)$. V tomto smyslu ztotožňujeme $\map T$ a $g$, a dualitu $\duality{f}{\map T} \mapsto \map T(f)$ ztotožňujeme s dualitou
\begin{equation}
    \duality{f}{g} \mapsto \int_\Omega f \conj g, \quad f \in L^p, \, g \in L^q \: .
    \label{eq:4.Dualita Lebesgue}
\end{equation}
Povšimněme si, že pokud v předchozí větě položíme $p=q=2$, získáme vlastnost $L^2(\Omega)' \simeq L^2(\Omega)$ a dualita \eqref{eq:4.Dualita Lebesgue} má stejný tvar jako skalární součin na $L^2(\Omega), \; \duality{f}{g} \equiv \innerprod{f}{g}_{L^2}$. Je přirozené se ptát, zda se za tímto výsledkem skrývá něco hlubšího. Odpověď je pozitivní.

\begin{theorem}[Rieszova-Fréchetova o reprezentaci] \label{4.Riesz-Frechet}
Buď $H$ Hilbertův prostor se skalárním součinem $(\cdot \, , \cdot)_H$, $\map T \in H'$.
Pak existuje právě jeden prvek $f \in H $ takový, že plátí následující:
\begin{enumerate}
    \item $\map T(x) = \innerprod{x}{f}_H \quad \text{pro každé } x \in H \:, $
    \item $\norm{\map T}_{H'} = \norm{f}_H \:.$
\end{enumerate}
\end{theorem}
\begin{proof}
viz Lukeš 2.9
\end{proof}

\begin{corollary}
Zobrazení $\map T \mapsto f$ je izometrický izomorfismus (zachovává normu a je bijektivní). Proto pro všechny Hilbertovy prostory $H$ můžeme provést ztotožnění $H' \simeq H$.
\end{corollary}

% ALERT ALERT TOHLE JE V POZNÁMKÁCH ŠPATNĚ
% T(x) BY NEBYLO LINEÁRÍ, ALE ANTILINEÁRNÍ ZOBRAZENÍ
%\begin{remark}
%V prvním tvrzení věty \ref{4.Riesz-Frechet} by mohlo ekvivalentně být $\map T(x) = \innerprod{f}{x}_H$. Skutečně: položme $\map S(x) = \conj{\map T(x)}$, pak podle Riesz-Fréchetovy věty nalezneme $g \in H$ takové, že $\map S(x) = \innerprod{x}{g}_H$. Potom $\map T(x) = \conj{\map S(x)} = \conj{\innerprod{x}{g}} = \innerprod{g}{x}$.
%\end{remark}

\begin{lemma}[O inkluzi duálních prostorů]
    Nechť jsou $X,Y$ Banachovy a platí $X \subset Y$ a $\norm{x}_Y = \norm{x}_X \; \forall x \in X$. Potom platí $Y' \subset X'$ ve smyslu zúžení zobrazení (restrikce). Zde je však důležité dát velký pozor, je snadné tento výsledek špatně interpretovat!
\end{lemma}
\begin{proof}
    \begin{align*}
    \text{Mějme } \map T \in Y'
    &\implies
    \map T \text{ spojité a lineární (na prvcích z $Y$) }
    \\
    &\implies
    \map T|_X \text{ spojité a lineární (na prvcích z $X$)}
    \implies
    \map T|_X \in X'
    \end{align*}
\end{proof}

\begin{example}
Bezhlavá aplikace předchozí inkluze nás může zahnat do slepých ulic.
\\
Uvažujme prostor $\R \subset \R^2$. Dle předchozího tvrzení můžeme prohlásit $(\R^2)' \subset (\R)'$. Protože jsou oba prostory Hilbertovy, lze je ztotožnit $(\R^n)' = \R^n $, proto by mělo platit $\R^2 \subset \R$, což je zřejmě nesmysl. Kde je ale chyba?

\begin{align*}
    \hspace{-3em}
    \R \subset \R^2 \quad \implies \quad
    (\R^2)' &\subset \R' \\[-5pt]
    \mask{(\R^2)'}{\rotatebox[origin=c]{90}{=}} &\ph{\subset} \;\; \mask{\R'}{\rotatebox[origin=c]{90}{=}} \\[-5pt]\mask{(\R^2)'}{\R^2} &\subset \mask{\R'}{\R}
\end{align*}
V předchozích úvahách jsme učinili dvě chyby: jednu větší, ️jednu menší. \begin{enumerate}
    \item Menší chybu jsme učinili tím, že jsme $(\R^n)' \simeq \R^n$ považovali rovnost, ve skutečnosti je to \textit{ztotožnění}. Každé lineární zobrazení na $\R^n$ má tvar $$\map T(x) = \sum_{j=1}^n \alpha_j x_j$$ a ztotožňuje se s $n$-ticí koeficientů $$T \simeq (\alpha_1, \dots \alpha_n) \in \R^n \text{ reprezentuje } (\R^n)' \: .$$ Na ono reprezentující $\R^n$ je tedy třeba nahlížet opravdu jako na prostor prvků, které reprezentují lineární zobrazení.
    \item Velkou chybu jsme učinili, když jsme inkluzi $(\R^2)' \subset \R'$ považovali za množinovou inkluzi. Tvrzení je třeba chápat v tomto smyslu:
    
    \begin{center}
    „Všechna lineární zobrazení pracující na $\R^2$ lze zúžit tak, aby pracovala na $\R$.“
    \end{center}
    
    Pokud je lineární zobrazení $\map T \in (\R^2)'$ reprezentovatelné dvojicí $(\alpha_1, \alpha_2) \in \R^2$, lze toto zobrazení skutečně zúžit na $\map T|_\R$ reprezentované $(\alpha_1, 0)$, který můžeme považovat za prostor $\R$. To je pravý smysl \uv{inkluze} $Y' \subset X'$.
\end{enumerate}
\end{example}

\bigskip

\begin{remark}
\uv{Duálnost} se často projevuje tím, že vzorce obsahující prvky $X$ a $X'$ vykazují jisté symetrie. Například víme, že
$$\norm{\map T}_{X'} = \sup_{\norm{x}_X \leq 1} |\map T(x)|.$$
Dále víme, že $|\map T(x)| \leq \norm{\map T} \norm{x}_X$. 
\end{remark}

Uvažujme nyní funkcionál $\map T \in X'$ takový, že $\norm{\map T}_{X'} \leq 1$. Pak jistě platí $|\map T x| \leq \norm{x}_X$ a po přechodu k supremu i \begin{align*}
    \sup_{\norm{\map T}_{X'} \leq 1} |\map T(x)| \leq \norm{x}_X \:.
\end{align*}
Následující věta ukazuje, že platí dokonce \textit{rovnost}: \begin{align*}
    \sup_{\norm{\map T}_{X'} \leq 1} |\map T(x)| = \norm{x}_X \:.
\end{align*}

\begin{theorem}[Hahnova-Banachova]
Buď $X$ Banachův, $x \in X$ takové, že $x \neq 0$. Pak existuje $\map T \in X'$ s vlastnostmi \begin{align*}
    \map T(x) = \norm{x}_X \:, \qquad \norm{\map T}_{X'} = 1 \:.
\end{align*}
\end{theorem}

\subsection{Duální zobrazení, duální operátor}

\begin{definition}[Duální zobrazení]
Nechť $X,Y$ jsou Banachovy prostory, $\map T \in \lin (X,Y)$, $\map T' : Y' \mapsto X'$. Řekneme, že $\map T'$ je duální zobrazení k $\map T$, jestliže \begin{align*}
    \map T' \circ \map y' = \map y' \circ \map T' \quad \text{pro všechna } \map y' \in Y' \:,
\end{align*}
neboli \begin{align*}
    \underbrace{(\map T' \circ \map y')}_{\in X'} \underbrace{(x)}_{\in X} = \underbrace{\map y'}_{\in Y'} \underbrace{(\map T x)}_{\in Y} \quad \text{pro všechna } \map y' \in Y' \text{ a všechna } x \in X \:.
\end{align*}
\end{definition}

V předchozí definici je klíčové uvědomit si příslušnost jednotlivých objektů. \begin{itemize}
    \item $\map y' \in Y'$ je zobrazení pracující na $Y$,
    \item $\map T' \map y' \in X'$ je zobrazení pracující na $X$,
    \item $(\map T \map y')(x)$ je objekt, který přiřazuje prvkům z $X \times X'$ číslo. To odpovídá struktuře duality.
\end{itemize}
S použitím tzv. \textit{kanonické} duality $\duality{\map F}{g} = \map F(g)$ můžeme zapsat definici duálního zobrazení v symetrickém tvaru:
\begin{align*}
    \duality{\map T' \map y'}{x} = \duality{\map y'}{\map T x} \: .
\end{align*}
Povšimněme si, že na levé straně máme zobrazení $X' \times X \to \C$ a na pravé zobrazení $Y' \times Y \to \C$.

\begin{lemma}
Je-li $\map T \in \lin (X,Y)$, pak i $\map T' \in \lin(Y', X')$. 
\end{lemma}
\begin{proof}
Linearita je zřejmá. Ukažme spojitost.  Zafixujme $\sequence{\map y'_n}{n} \subset Y'$ takovou, že $\map y'_n \rightarrow \map y'$. Ukážeme, že posloupnost $\sequence{\map T \map y}{n} \subset X'$ konverguje k $\map T \map y_n$. Platí
\begin{align*}
    \norm{\map T' \map y'_n - \map T' \map y'}_{X'} 
    =&
    \sup_{\norm{x} \leq 1} \norm{\map T' \map y'_n (x)- \map T' \map y' (x)}_X 
    =
    \sup_{\norm{x} \leq 1} \norm{\map y'_n  (\map T x)- \map y' (\map T x)}_X 
    =
    \sup_{\norm{x} \leq 1} \norm{(\map y'_n-\map y') (\map T x)}_X 
    \leq \\
    \leq&
     \sup_{\norm{x} \leq 1} \norm{\map y'_n - \map y'} \norm{\map T} \norm{x}
    =
    \norm{\map y'_n -\map y'} \norm{\map T} \rightarrow 0\:,
\end{align*}
což jsme chtěli ukázat.
\end{proof}
\begin{remark}
Dá se ukázat, že: $\map T$ je kompaktní operátor právě tehdy, když je $\map T'$ kompaktní operátor (tzv. Schauderova věta). Sami zkuste dokázat, že $\norm{\map T} = \norm{\map T'}$.
\end{remark}

\bigskip

Dále nás bude zajímat, lze-li ztotožnit $\map T$ a $\map T'$ (podobně jako ztotožňujeme Hilbertův prostor s vlastním duálem). To by znamenalo:
$$
    \underbrace{\map T}_{X \to Y} \;\simeq \underbrace{\map T'}_{Y' \to X'}
$$
Muselo by tedy platit něco ve smyslu
$$ X \simeq Y' \quad \text{a} \quad Y \simeq X' \: , $$
vezmeme-li nyní duál předchozích výrazů, dostaneme
$$ X' \simeq Y'' \quad \text{a} \quad Y' \simeq X'' \: . $$
Zkombinováním obou „rovnic“ získáme s trochou \textit{máchání rukami} výraz
$$ X \simeq X'' \quad \text{a} \quad Y \simeq Y'' \: . $$
To by mohlo platit pro Hilbertovy prostory, kde je dokonce už i $X' \simeq X$. Obecně k danému $\map T$ nemusí vždy existovat $\map T'$, výše uvedené vlastnosti platí pouze v případě, že existuje. Ale v Hilbertově prostoru je opět vše lepší:

\begin{theorem}[O duálním zobrazení mezi Hilbertovými prostory]
Nechť $H_1, H_2$ jsou Hilbertovy prostory a $\map T \in \lin(H_1, H_2)$. Pak existuje právě jedno zobrazení $\map T' : H_2 \mapsto H_1$ takové, že 
\begin{align} \label{eq4.dualHilbert}
    (\map T x, y)_{H_2} = (x, \map T'y)_{H_1} \quad \text{pro všechna } x \in H_1, \; y \in H_2 \:.
\end{align}
Pro takové zobrazení navíc platí: \begin{enumerate}
    \item $\map T' \in \lin(H_2, H_1)$,
    \item $\norm{\map T'} = \norm{\map T \vph{\map T'}}$.
\end{enumerate}
\end{theorem}

\begin{remark}
Aplikujeme-li komplexní sdružení na rovnici \eqref{eq4.dualHilbert}, dostaneme \begin{align*}
    \conj{\innerprod{\map T x}{y}}_{H_2} = \conj{\innerprod{x}{\map T' y}}_{H_1} \rightarrow \innerprod{\map T' y}{x}_{H_1} = \innerprod{y}{\map T x}_{H_2} \:.
\end{align*}
Na Hilbertových prostorech tedy hraje roli duality skalární součin.
\end{remark}
\begin{proof}
Zafixujme $y \in H_2$ a definujme
\begin{gather*}
    \map L_y \in \lin(X, \C) \: ,\\
    \map L_y(x) \coloneqq \innerprod{\map T x}{y}_{H_2} \: .
\end{gather*}
Podle Rieszovy-Fréchetovy věty (Věta \ref{4.Riesz-Frechet}) existuje právě jedno $z \in H_1$ takové, že \begin{align*}
    \map L_y(x) = \innerprod{x}{z}_{H_1} \:.
\end{align*}
Nyní definujeme zobrazení:
\begin{gather*}
    \map T': H_2 \to H_1 \: ,\\
    \map T'(y) \coloneqq  z \: .
\end{gather*}
Toto zobrazení má vlastnost
$$
    (\map T x, y)_{H_2} = (x,\map T' y)_{H_1} \quad \text{pro libovolné } x\in H_1, y \in H_2 \:.
$$
Tím jsme dokázali první část tvrzení o existenci a jednoznačnosti zobrazení $\map T'$.

Ukažme jeho linearitu. Zřejmě je pro každé $x \in H_1$ splněno
\begin{align*}
    \left( \map T' (\alpha y_1 + \beta y_2) , x \right)_{H_1} =& \left( \alpha y_1 + \beta y_2 , \map T x \right)_{H_2} = \alpha (y_1, \map T x)_{H_2} + \beta (y_2, \map T x)_{H_2} = \\
    =&
    \alpha (\map T' y_1, x)_{H_1} + \beta (\map T' y_2, x)_{H_1} = (\alpha \map T' y_1 + \beta \map T' y_2, x)_{H_1}
\end{align*}
a odtud
\begin{align*}
    \map T' (\alpha y_1 + \beta y_2) = \alpha \map T' y_1 + \beta \map T' y_2 \:.
\end{align*}

Ukažme spojitost zobrazení $\map T'$ pomocí omezenosti jeho normy. Podle definice zobrazení $\map T'$ a druhé části Rieszovy-Fréchetovy věty (Věta \ref{4.Riesz-Frechet}) dostáváme \begin{align} \label{eq:4.Vdual1}
    \norm{\map T' y}_{H_1} = \norm{ z}_{H_1} = \norm{\map L_y} \:.
\end{align}
Dále je podle definice $\map L_y$ a Cauchyovy-Schwarzovy nerovnosti \begin{align*}
    \norm{\map L_y x} = |(\map Tx, y)_{H_2}| \leq \norm{\map Tx}_{H_2} \norm{y}_{H_2} \leq \norm{\map T} \norm{x}_{H_1} \norm{y}_{H_2}
\end{align*}
a odtud s použitím \eqref{eq:4.Vdual1} plyne
\begin{align*}
    \norm{\map T' y}_{H_1} = \norm{\map L_y} = \sup_{\norm{x}_{H_1} \leq 1} \norm{\map L_y x}_{H_2} \leq \norm{\map T} \norm{y}_{H_2} \:.
\end{align*}
\begin{align*}
    \norm{\map T'} = \sup_{\norm{y}_Y \leq 1} \norm{\map T' y} \leq  \norm{\map T} \sup_{\norm{y} \leq 1} \norm{y}_{H_2} = \norm{\map T} < + \infty \:.
\end{align*}
Tím jsme ověřili spojitost a celkově $\map T' \in \lin(H_2, H_1)$.

Zbývá ukázat rovnost norem $\norm{\map T'} = \norm{\map T}$. Za tím účelem definujeme $\map T'' = (\map T')'$. O tomto zobrazení již víme, že $\map T'' \in \lin(H_1, H_2)$, $(\map T'' x,y)_{H_2} = (x, \map T' y)_{H_1}$ a $\norm{\map T''} \leq \norm{\map T'}$. Pak s použitím předchozích částí důkazu dostaneme \begin{align*}
    (\map T'' x,y)_{H_2} = (x, \map T'y)_{H_1} = \conj{(\map T' y, x)}_{H_1} = \conj{(y, \map T x)}_{H_2} = (\map T x,y)_{H_2} \quad \text{ pro všechna } x \in H_1, y \in H_2 \:,
\end{align*}
tedy \begin{align*}
    \map T'' x= \map T x \quad \text{pro každé } x \in H_1 \:.
\end{align*}
Odtud \begin{align*}
    \norm{\map T} = \norm{\map T''} \leq \norm{\map T'} \leq \norm{\map T} \:,
\end{align*}
a proto musí platit v předchozím řetězci všude rovnosti.
\end{proof}
\begin{definition}[Hermitovsky sdružený operátor]
Nechť $H_1, H_2$ jsou Hilbertovy prostory, $\map T \in \lin (H_1, H_2) $. Zobrazení $\map T'$ s vlastnostmi z předchozí věty nazýváme hermitovsky sdružený operátor s $\map T$ (případně adjungovaný operátor~k~$\map T$).
\end{definition}

\begin{definition}[Samoadjungovaný (omezený) operátor.]
Nechť $H$ je Hilbertův prostor. Operátor $\map T \in \lin(H)$ nazveme (omezený) samoadjungovaný  (případně hermitovský), jestliže $\map T = \map T'$.
\end{definition}
\begin{remark}
V předchozí definici jsou oba operátory $\map T, \map T'$ definovány na celém Hilbertově prostoru. Zdůrazňujeme zde, že mluvíme o \uv{omezených samoadjungovaných} operátorech. V případě neomezených operátorů uvidíme, že definiční obor operátorů zcela mění spektrální vlastnosti a definice hermitovskosti a samoadjungovanosti je složitější.
\end{remark}

\subsection{Vlastnosti samoadjungovaných operátorů}

\begin{lemma}[O vlastních číslech samoadjungovaných operátorů]
Nechť $\map T \in \lin (H)$ je samoadjungovaný. Pak má pouze reálná vlastní čísla a vlastní vektory příslušející různým vlastním číslům jsou na sebe kolmé.
\end{lemma}
\begin{proof}
První část tvrzení plyne z rovnosti \begin{align*}
\lambda \norm{x}^2_H = (\lambda x, x)_H = (\map T x,x)_H = (x, \map T x)_H = (x, \lambda x)_H = \conj{\lambda} (x,x)_H = \conj{\lambda} \norm{x}_H ^2 \:.
\end{align*}
Nechť jsou $\lambda_1 \neq \lambda_2$ vlastní čísla příslušející vlastním vektorům $y_1,y_2$. Pak z rovnosti \begin{align*}
    (\lambda_1 - \lambda_2)(y_1,y_2)_H = (\lambda_1 y_1,y_2)_H - (y_1, \lambda_2 y_2)_H = (\map T y_1, y_2)_H - (y_1, \map T y_2)_H = (\map T y_1,y_2)_H - (\map T y_1,y_2)_H = 0
\end{align*}
vidíme, že musí platit $(y_1,y_2)=0$.
\end{proof}
\begin{remark}
Samoadjungovaný operátor může mít ovšem i jiné prvky spektra. O nich obecně nevíme nic.
\end{remark}
\begin{theorem}[O spektrálních vlastnostech samoadjungovaných operátorů]
Nechť $H$ je Hilbertův prostor a $\map T \in \lin (H)$ je samoadjungovaný operátor. Označme \begin{align*}
    m(\map T) = \inf \left \lbrace (\map T x,x)_H : \norm{x}_H = 1 \right \rbrace \qquad M(\map T) = \sup \left \lbrace (\map T x,x)_H : \norm{x}_H = 1 \right \rbrace
\end{align*} Pak \begin{enumerate}
    \item Je-li $\lambda \in \sigma(\map T)$, pak $\lambda \in [m(\map T), M(\map T)]$.
    \item Platí $\rho(\map T) = \norm{\map T}$. Speciálně: alespoň jedna z hodnot $\lambda = \pm \norm{\map T}$ je vlastním číslem $\map T$.
\end{enumerate}
\end{theorem} 

\begin{exercise}
Nechť $\mathbb{A} \in \R^{n \times n}$ je reálná symetrická matice, $x \in \R^n$ a $f(x) = (\mathbb{A} x,x) = \sum_{i,j} a_{ij} x_i x_j$. Určete minimum a maximum $f$ za podmínky $\norm{x} =1$.
\end{exercise}

\subsection{Kompaktní samoadjungované operátory na Hilbertových prostorech}

Nejprve zopakujeme výsledky, které zatím máme o kompaktních samoadjungovaných operátorech. Na Hilbertově prostoru $H$ uvažujme $\map K \in \comp (H)$ takový, že $(\map K x,y)_H = (x, \map K y)_H$ pro všechna $x,y \in H$. Pak platí \begin{enumerate}
    \item $\map K$ má nejvýše spočetně mnoho vlastních čísel, všechna vlastní čísla jsou reálná.
    \item Jediným dalším prvkem spektra $\sigma(\map K)$ může být číslo $0$, ta může nebo nemusí být vlastním číslem.
    \item Ke každému nenulovému vlastnímu číslu existuje nejvýše konečně mnoho lineárně nezávislých vlastních vektorů. Navíc vlastní vektory odpovídající různým vlastním číslům jsou na sebe vždy kolmé.
\end{enumerate}

Hlavním výsledkem této kapitoly bude Hilbertova-Schmidtova věta, která říká následující: všechny vlastní vektory odpovídající všem vlastním číslům kompaktního samoadjungovaného operátoru tvoří bázi Hilbertova prostoru. Nejprve je však potřeba provést přípravné práce.

První takovou prací je připomenutí ortogonálního doplňku z lineární algebry.
\begin{definition}[Direktní součet podprostorů]
Nechť $H$ je lineární vektorový prostor a $A,B$ jsou podprostory $H$. Řekneme, že $H$ je direktním součtem $A$ a $B$ (píšeme $A \oplus B = H$, jestliže \begin{enumerate}
    \item $A+B=H$, tj. pro každé $h \in H$ existují $a \in A$, $b \in B$ takové, že $a+b=h$,
    \item $A \cap B = \{0 \}$.
\end{enumerate}
\end{definition}

\begin{example}
Jsou-li $A$ a $B$ dvě různé přímky protínající počátek, pak lze psát $\R^2 = A \oplus B$.
\end{example}

\begin{definition}[Ortogonální doplněk]
Nechť $H$ je Hilbertův prostor a $A$ je uzavřený lineární podprostor v $H$. Definujeme ortogonální doplněk $A^\bot \subset H$ předpisem \begin{align*}
    A^\bot := \left \lbrace y \in H: (x,y)=0 \text{ pro každé } x \in A  \right \rbrace \:.
\end{align*}
\end{definition}

\begin{theorem}[O vlastnostech ortogonálního doplňku]
Nechť $H$ je Hilbertův prostor a $A$ je uzavřený lineární podprostor $H$ a $A^\bot$ je jeho ortogonální doplněk. Pak \begin{enumerate}
    \item $A^\bot$ je uzavřený lineární podprostor $H$,
    \item platí $ (A^\bot)^\bot = A$,
    \item platí $A \oplus A^\bot = H$.
\end{enumerate}
\end{theorem}
\begin{proof}
\begin{enumerate}
    \item Linearita je zřejmá. Uzavřenost plyne ze spojitosti skalárního součinu. Pro konvergentní posloupnost $ \sequence{y_n}{n} : y_n \rightarrow y$ totiž máme \begin{align*}
        0 = \innerprod{x}{y_n}_H \rightarrow \innerprod{x}{y}_H = 0 \:.
    \end{align*}
    \item Druhou část ponecháváme čtenáři jako jednoduché cvičení.
    \item Třetí část lze nejlépe nahlédnout v kontextu tvrzení o kolmé projekci.
    
    Je-li $A$ lineární podprostor $H$, pak pro každé $x \in H \setminus A$ existuje prvek $ Px \in A $ takový, že \begin{align*}
        \innerprod{x - Px}{y} = 0 \quad \text{pro všechna } y \in A \:,
    \end{align*}
    neboli $x-Px \in A^\bot$.
    
    Nyní už je tvrzení zřejmé. Pro $x \in A$ máme $x = x+ 0 $. Pro $x \in H \setminus A$ můžeme psát \begin{align*}
        x = \underbrace{(x -Px)}_{\in A} + \underbrace{Px}_{\in A^\bot} \:.
    \end{align*}
    Navíc pro $ v \in A \cap A^\bot $ máme $\innerprod{v}{v} = 0$, odtud $v =0$. Tím jsme dokázali direktnost součtu.
\end{enumerate}
\end{proof}

Druhou přípravnou prací je zopakování některých výsledků o Fourierových řadách.

Připomeňme, že o metrickém prostoru $X$ říkáme, že je separabilní, jestliže v něm existuje spočetná hustá množina. Ortogonálním systémem v $H$ nazýváme posloupnost $\sequence{e_n}{n} \subset H$ takovou, že pro $i\neq j$ platí $(e_i,e_j)_H =0$. O ortogonálním systému řekneme, že je úplný, právě když \begin{align*}
    y \in H, (y,e_n)_H =0 \implies y = 0 \:.
\end{align*}

\begin{theorem}[O Fourierových řadách na Hilbertově prostoru.] \label{4.fourier}
Nechť $H$ je Hilbertův prostor. Pak jsou následující tvrzení ekvivalentní: \begin{enumerate}
    \item $H$ je separabilní prostor,
    \item Existuje úplný spočetný ortogonální systém $\sequence{e_n}{n} \subset H$,
    \item Pro každé $x \in H$ platí \begin{align*}
        x = \sum_{n=1}^\infty \dfrac{(x,e_n)_H}{\norm{e_n}^2} e_n \:,
    \end{align*}
    \item Pro každé $x \in H$ platí \begin{align*}
        \norm{x}_H^2 = \sum_{n=1}^\infty \dfrac{(x,e_n)^2_H}{\norm{e_n}^2} \:. \qquad \text{(tzv. Parsevalova rovnost)}
    \end{align*}
\end{enumerate}
\end{theorem}

Nyní jsme připraveni dokázat klíčovou větu této kapitoly.
\begin{theorem}[Hilbertova-Schmidtova] \label{4.Hilbert-Schmidt}
Nechť $H$ je Hilbertův prostor, $\map T$ je kompaktní samoadjungovaný operátor na $H$. Označme $\Lambda$ uzavřený poprostor $H$ generovaný všemi vlastními vektory $\map T$, které odpovídají všem nenulovým vlastním číslům $\map T$. Pak \begin{align*}
    H = \Lambda \oplus \Ker \map T \:.
\end{align*}
\end{theorem}

\begin{proof}
Důkaz věty rozdělíme do několika kroků.

\underline{Krok 1:} Vlastnosti prostoru $\Lambda$. \\
Díky kompaktnosti a samoadjungovanosti $\map T$ existuje nejvýše spočetná posloupnost $\sequence{\lambda_j}{j}$ vlastních čísel. Označme podprostor příslušející $j$-tému vlastnímu číslu \begin{align*}
    E_j = \Ker \left( \map T - \lambda_j \map{Id} \right) = \lbrace x \in H : x \neq 0, \map T x = \lambda_j x \rbrace
\end{align*}
Díky kompaktnosti $\map T$ je $\dim E_j := n_j < + \infty$. Označme $B_j$ bázi prostoru $E_j$. Můžeme bez újmy na obecnosti předpokládat, že je tato báze ortogonalizovaná (jinak můžeme použít například Gramovu-Schmidtovu ortogonalizaci). Definujme \begin{align*}
    B = \bigcup_{j=1}^\infty B_j \:.
\end{align*}
Pak i $B$ je nejvýše spočetná množina a je tvořena vlastními vektory. Ukažme, že je i ortogonální množina: zvolme $x \neq y \in B$. Pak oba prvky buď patří do stejné $B_j$ a jsou na sebe kolmé díky předpokladu výše, nebo jsou z různých bází $B_i, B_j$ a jsou na sebe kolmé díky samoadjungovanosti $\map T$.

Definujme prostor všech konečných součtů prvků z $B$ předpisem \begin{align*}
    \Lambda_0 = \left\lbrace z \in H: z = \sum_{j=1}^n \alpha_j e_j, \alpha_j \in \mathbb{C} , e_j \in B \right\rbrace =  \mathrm{span}\, B
\end{align*}
a dále prostor všech nejvýše spočetných součtů prvků z $B$ předpisem \begin{align*}
    \Lambda := \conj{\Lambda_0} = \left\lbrace z \in H: z = \sum_{j=1}^\infty \alpha_j e_j, \alpha_j \in \mathbb{C} , e_j \in B \right\rbrace \:.
\end{align*}
Pak $\Lambda$ je uzavřený lineární podprostor $H$, je tedy Hilbertovým prostorem. 

Navíc je $\Lambda$ separabilní. Spočetnou hustou množinu v něm tvoří \begin{align*}
    \left\lbrace z \in H: z = \sum_{j=1}^N \beta_j e_j, \beta_j \in \Q , e_j \in B, N \in \N \right\rbrace
\end{align*}

\underline{Krok 2:} Příslušnost množin $\map T(\Lambda)$ a $\map T(\Lambda^\bot)$. \\
Nejprve ukážeme, že $\map T(\Lambda) \subset \Lambda$. Zvolme $x \in \Lambda$. Pak \begin{align*}
    \map T x = \map T \left(\sum_{j=1}^\infty \gamma_j e_j \right) = \sum_{j=1}^\infty \gamma_j \map T e_j = \sum_{j=1}^\infty \gamma_j \lambda_j e_j \:,
\end{align*}
tedy $\map T x \in \Lambda$. 

Dále ukažme, že $\map T(\Lambda^\bot) \subset \Lambda^\bot$. Zvolme $x \in \Lambda, y \in \Lambda^\bot$. Pak $\map T x \in \Lambda^\bot$ a z rovnosti \begin{align*}
    (\map T y,x)_H = (y, \map T x)_H =0
\end{align*}
vidíme, že musí nutně platit $\map T y \in \Lambda^\bot$. Tím jsme dokázali, že $\map T(\Lambda^\bot) \in \Lambda^{\bot}$. Celkově jsme ukázali, že  $\map T(\Lambda^\bot)$ je také uzavřený podprostor v $H$ a tedy Hilbertův.

\underline{Krok 3:} Platnost tvrzení $\map T(\Lambda^\bot) = \lbrace 0 \rbrace$. \\
Definujme zúžení operátoru $\map T$ předpisem \begin{align*}
    \map {\tilde T} := \map T\rvert_{\Lambda^\bot}
\end{align*}
Protože $\map T(\Lambda^\bot) \in \Lambda^{\bot}$, platí $\map{\tilde T} : \Lambda^{\bot} \mapsto \Lambda^{\bot}$. Tedy $\map{\tilde T}$ je také kompaktní a samoadjungovaný operátor na $\Lambda^{\bot}$.

Ukážeme sporem, že $\map{\tilde T}$ nemá žádné nenulové vlastní číslo. Nechť platí negace tohoto výroku a existuje vlastní číslo $\lambda \neq 0$ a vlastní vektor $y \in \Lambda^{\bot} , y \neq 0$ takové, že $ \map{\tilde T} y = \lambda y$. Díky samoadjungovanosti je pak \begin{align*}
    \map T y = \map{\tilde T} y = \lambda y
\end{align*}
a tedy $\lambda$ je i vlastní číslo operátoru $\map T$. Podle druhé části důkazu je ale potom $y \in \Lambda$. Tedy $y \in \Lambda \cap \Lambda^\bot$, což implikuje $y=0$ a to je ve sporu s předpoklady.

Celkově jsme zjistili, že $\map{\tilde T}$ je kompaktní operátor, který nemá nenulové vlastní číslo, tedy $\sigma(\map{\tilde T}) \subset \{ 0 \}$. Podle Věty o spektrálních vlastnostech kompaktních operátorů (Věta XY) je $\rho(\map{\tilde T}) = 0$, a tedy  $\norm{\map{\tilde T}} =0$, což je ekvivalentní s $\map{\tilde T} \equiv 0$. Z konstrukce $\map{\tilde T}$ pak snadno nahlédneme, že $\map T(\Lambda^\bot)= \{0 \}$.

\underline{Krok 4:} Dokončení důkazu. \\
Předchozí část důkazu nám dává $\Lambda^\bot \subset \Ker \map T$. Zjevně platí $\Lambda \oplus \Lambda^\bot = H$, tedy \begin{align*}
    \Lambda + \Ker \map T = H \:.
\end{align*}
Zbývá ukázat, že $\Lambda \cap \Ker \map T = \{0 \}$, tím dokážeme direktnost součtu těchto prostorů.

Zvolme $z \in \Lambda \cap \Ker \map T$. Díky jeho příslušnosti do $\Lambda$ lze psát \begin{align*}
    0 = \map T z = \map T \left( \sum_{j=1}^\infty \alpha_j e_j \right) =  \sum_{j=1}^\infty \alpha_j \map T e_j =  \sum_{j=1}^\infty \alpha_j \lambda_j e_j \:.
\end{align*}
To je Fourierova řada nulového prvku. Z teorie jednoznačnosti Fourierových řad pak vyplývá, že \begin{align*}
    \alpha_j \lambda_j =0 \quad \text{pro všechna } j \in \N \:.
\end{align*}
Díky nenulovosti $\lambda_j$ dostáváme \begin{align}
    \alpha_j =0 \quad \text{pro všechna } j \in \N
\end{align}
a tedy $z = 0$, což jsme chtěli ukázat.
\end{proof}

\begin{remark}
\begin{enumerate}
    \item První část důkazu se vlastně netýká věty samotné, jen osvětluje vlastnosti prostoru $\Lambda$.
    \item V důkazu nám stačilo pouze ukázat, že $\Lambda^\bot \subset \ker \map T$. Dá se dokonce ukázat, že $\Lambda^\bot = \ker \map T$.
    
    Zvolme $z \in \ker \map T \setminus \Lambda^\bot $ takové, že $z \neq 0$. Pak existuje $n \in \N$ takové, že $\innerprod{z}{e_n} \neq 0$. (Skutečně, jinak by platilo $\innerprod{z}{\sum_{j=1}^\infty \beta_n e_n} =0 $ a odtud $z \in \Lambda^\bot$.) Pak ale platí $\map T z =0$, neboť $z \in \Ker \map T$. Odtud dostáváme pro všechna $n \n \N$ \begin{align*}
        0 = \innerprod{\map T z}{e_n} = \innerprod{z}{\map T e_n} = \innerprod{z}{\lambda_n e_n} = \lambda_n \innerprod{z}{e_n} \:,
    \end{align*}
    což je ve sporu s $\lambda_n \neq 0$. Tím jsme ověřili, že $\Lambda^\bot \subset \ker \map T$.
\end{enumerate}
\end{remark}

\begin{corollary}[O rozkladu Hilbertova prostoru pomocí kompaktního samoadjungovaného operátoru] \label{4.rozklad I}
Nechť $H$ je Hilbertův prostor, $\map T$ je kompaktní samoadjungovaný operátor na $H$, $\sequence{e_j}{j} \subset H$ je ortogonální systém všech vlastních vektorů příslušející všem nenulovým vlastním číslům operátoru $\map T$. Pak pro každý prvek $h \in H$ existuje posloupnost $\sequence{\alpha_j}{j}$ taková, že \begin{align} \label{eq:4.rozklad h}
    h = \sum_{j=1}^\infty \alpha_j e_j + z \:,
\end{align}
přičemž $z \in \Ker \map T$. Dále \begin{align} \label{eq:4.rozklad Th}
    \map T h = \sum_{j=1}^\infty \alpha_j \lambda_j e_j
\end{align}
a platí \begin{align*}
    \alpha_k = \dfrac{(h, e_k)}{\norm{e_k}^2} \quad \text{pro všechna } k \in \mathbb{N} \:.
\end{align*}
\end{corollary}

\begin{remark}
Hilbertova-Schmidtova věta \ref{4.Hilbert-Schmidt} a důsledek \ref{4.rozklad I} rozšiřuje známou větu ze čtvrtého semestru matematické analýzy (Věta \ref{4.fourier}) i na Hilbertovy prostory, které nejsou separabilní. Neseparabilní část prostoru je právě $\Ker \map T$. Zatímco se v rovnici \eqref{eq:4.rozklad h} objevuje i tato neseparabilní část, v rovnici \eqref{eq:4.rozklad Th} už nevystupuje. Tato vlastnost pak umožňuje využívat vlastností rozkladu do Fourierových řad u kompaktních samoadjungovaných operátorů i na neseparabilních prostorech.
\end{remark}

Na závěr kapitoly se rozloučíme s větou, která (mimo jiné) umožňuje konstruovat kompaktní operátory na Hilbertových prostorech.

\begin{theorem}[O rozkladu Hilbertova prostoru pomocí omezené posloupnosti] \label{4.rozklad II}
Nechť $H$ je separabilní Hilbertův prostor, $\sequence{e_j}{j} \subset H$ je úplný ortonormální systém a $\sequence{\alpha_j}{j} \subset \C$ je omezená posloupnost. Označme \begin{align*}
    M:= \sup_{j \in \N} \{ |\alpha_j|\} < + \infty \:.
\end{align*}
Definujme operátor $\map T$ předpisem \begin{align*}
    \map T h = \sum_{j=1}^\infty (h, e_j)_H \cdot \alpha_j  e_j \:.
\end{align*}
Pak \begin{enumerate}
    \item  $\map T$ je dobře definován na celém $H$, $\map T \in \lin (H)$ a $\norm{\map T} = M$.
    \item Platí\begin{align*}
        \map T \text{ je samoadjungovaný } \Longleftrightarrow \alpha_j \in \mathbb{R} \text{ pro všechna } j \in \N \:. 
    \end{align*}
    \item Platí \begin{align*}
        \map T \text{ je kompaktní } \Longleftrightarrow \text{ existuje přerovnání poslopnosti} \sequence{\alpha_j}{j} \text{ tak, že } \lim_{n \rightarrow \infty} \alpha_n = 0 \:.
    \end{align*}
\end{enumerate}
\end{theorem}

\begin{example}
\begin{enumerate}
    \item Volme posloupnost $\alpha_n=1$. Pak operátor generovaný takovou posloupností je zřejmě $\map{Id}$. Podle druhé a třetí části předchozí věty je $\map{Id}$ samoadjungovaná, ale není kompaktní.
    \item Volme posloupnost $\alpha_n = \dfrac{1}{n}$. Pak podle předchozí věty generuje tato posloupnost operátor, který je samoadjungovaný a navíc díky $\alpha_n \rightarrow 0$ je kompaktní.
\end{enumerate}
\end{example}

\begin{remark}
V kvantové statistické mechanice se pracuje s tzv. operátorem hustoty $\hat{\rho}$ definovaným předpisem \begin{align*}
    \hat{\rho} := \sum_{j=1}^\infty p_j \lvert \psi_j \rangle \langle \psi_j \rvert \:, \quad \text{kde } \sequence{\lvert \psi_j \rangle}{j} \in H \text{ jsou takové, že } |\langle \psi_j \lvert \psi_j \rangle |=1 \text{ a } \sum_{j=1}^\infty p_j =1 \:.
\end{align*}
Povšimněme si, že v nediracovské notaci bychom takový operátor zapsali předpisem \begin{align*}
    \boldsymbol \rho (h) = \sum_{j=1}^\infty p_j (\psi_j,h)_H \psi_j \:.
\end{align*}
Ke splnění podmínek z předchozí věty mu chybí jen to, že $\sequence{\lvert \psi_j \rangle}{j}$ nemusí (a většinou netvoří) ortonormální posloupnost. Nicméně se dá ukázat, že operátor hustoty je vždy kompaktní a samoadjungovaný. Dokonce patří do speciální třídy kompaktních operátorů, pro které lze definovat tzv. stopu operátoru předpisem \begin{align*}
    \mathrm{Tr}\, \hat \rho := \sum_{j=1}^\infty \langle \psi_j \rvert \hat \rho \lvert \psi_j \rangle
\end{align*}
(tzv. \textit{trace-class} operátory).
Více lze najít například v knize: \textsc{BLANK, Jiří, Pavel EXNER a Miloslav HAVLÍČEK. Hilbert Space Operators in Quantum Physics Theoretical and Mathematical Physics. Melville, N.Y.: Springer, 2008. ISBN 9781402088698.}

\end{remark}

