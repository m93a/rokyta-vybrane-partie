\section{Operátorová trivia}

Co budeme považovat za známé:
\begin{itemize}
    \item \uu{Vektorový prostor} $X$ nad\footnote{%
    Zde $X$ je množina a~$\K$ je těleso.}
    $\underbrace{\R\text{ nebo }\C}_{\text{skaláry}}$ (ale existují i skaláry z $\mathbb{Q}$). Tam, kde nebude důležité, jestli jde o~$\R$ nebo o~$\C$, budeme někdy používat značení $\K$ (znamenající tedy „buď $\R$ nebo $\C$“). Kromě termínu „vektorový prostor“ (VP) se používá i termín „lineární prostor“ (LP), případně „lineární vektorový prostor“ (LVP).
    
    \item \uu{Lineárně nezávislá} (LN) \uu{množina} ve VP: $M \subseteq X$ je LN, pokud $a_1 x_1 + \dots + a_n x_n = 0 \implies a_1 = a_2 = \dots = a_n = 0$ pro všechny možné n-tice $(x_1, \dots x_n) \subseteq M$ a všechny skaláry $a_j \in \K$.
    %V poznámkách na webu \subseteq a ne \subset. -J
    \\[5pt]
    \Poznamka I v případě, že $M$ je nekonečná, uvažujeme pouze konečné součty (tj. všechny možné „libovolně dlouhé, ale konečné“ součty). Je totiž potřeba si uvědomit, že v~obecném VP není definován pojem konvergence, a~tedy samotný pojem nekonečného součtu nemá v~obecném VP smysl.
    \\[5pt]
    „Konečné součty patří do algebry, nekonečné do analýzy.“
    
    \item \uu{Báze $X$} ($X \neq \emptyset$, $X \neq \{0\}$):
    \begin{enumerate}
        \item Pokud existuje \uu{konečná} LN množina $B$ v~$X$ taková, že její lineární obal
        $$ \Lin(B) \coloneqq \big\{ \sum_{j=1}^n a_j x_j, \;  x_j \in B, a_j \in \K, n \in \N \big\} $$
        je roven $X$ (říkáme, že $B$ \uu{generuje} $X$), pak takovou množinu nazvu \uu{bází $X$}.
        Její mohutnost je pak určena jednoznačně (lze ukázat),tomuto číslu pak říkáme \uu{dimenze $X$}: $\dim X \coloneqq \operatorname{moh}(B) \in \N$.
        \item Pokud v $X$ $\forall n \in \N$ existuje LN množina o $n$ prvcích, říkáme, že $\dim X = \infty$. V tomto případě je pojem báze striktnější: bází $X$ nad $\K$ je v tomto případě taková \uu{nekonečná množina} $B$, která splňuje:
        \begin{enumerate}
            \item $B$ je LN (ve smyslu všech konečných lin. kombinací – viz výše).
            \item $\forall x \in X \; \exists n(x) \in \N$ a odpovídající \uu{konečný počet} prvků báze $x_1, \dots x_{n(x)}$ a koeficientů $a_j \in \K$, že
            $$ x = \sum_{j=1}^{n(x)} a_j x_j \,. $$
        \end{enumerate}
        
        \Poznamka I zde tedy jde principiálně o konečné součty prvků, vybíraných z nekonečné množiny (pro různá $x$ může jít o různé sady prvků báze). Této nekonečné bázi se říká \uu{Hammelova báze $X$ nad $\K$}. Otázka zní, zda každý VP $X$ (který není konečné dimenze) má Hammelovu bázi. Odpověď \textit{ano} je důsledkem \uu{axiomu výběru} (kdo jej tedy neuznává, pro něj by odpověď byla \textit{ne}).
    \end{enumerate}
\end{itemize}

\Priklad{
    $\underbrace{\R}_{\text{VP}}$ nad $\underbrace{\R}_{\text{skaláry}}$ má dimenzi 1: $\forall x \in \R \exists a = x \in \R$, že $x = a \cdot 1$. Báze je tedy $\{ 1 \}$.
    \\ \\
    $\underbrace{\R^n}_{\text{VP}}$ nad $\underbrace{\R}_{\text{skaláry}}$ má dimenzi $n$.
    \\ \\
    $\underbrace{\R}_{\text{VP}}$ nad $\underbrace{\Q}_{\text{skaláry}}$ má dimenzi $\infty$: totiž každá konečná báze reálných čísel generuje pomocí spočetné množiny koeficientů ($\Q$) jen spočetně mnoho prvků, a $\R$ je nespočetná.

    
    Tím je vyřešen problém dimenze $\R$ nad $\Q$. Všimněte si, že dimenzi $\infty$ lze určit i bez znalosti odpovědi na otázku existence báze, tj. bez nutnosti axiomu výběru. Pokud však připustíme axiom výběru, pak existuje Hammelova báze $\R$ nad $\Q$, tj. $\exists B \subset \R$, že $B$ je LN ve smyslu výše uvedené definice a~$
    \forall x \in \R \;
    \exists n(x) \in \N \;
    \exists b_1, \dots b_{n(x)} \in B \;
    \exists q_1, \dots q_{n(x)} \in \Q
    $,
    že $x = \sum_{j=1}^{n(x)} q_j b_j$.
    Rozmyslete si, že B je nutně nespočetná (jinak vygeneruji jen spočetně mnoho prvků).
    }


\begin{itemize}
    \item \uu{Norma na LP}: \hspace{1em}
    $\norm{\cdot}: X \to \R$, že \parbox[t]{20em}{
        $\norm{x+y} \leq \norm{x} + \norm{y}$, \\[3pt]
        $\norm{ax} = |a| \cdot \norm{x}$, \\[3pt]
        $\norm{x} = 0 \iff x = 0$.
    }
    
    $(X, \norm{\cdot})$ je potom normovaný lineární prostor (NLP). V něm lze definovat konvergenci:
    $$
        x_n \xrightarrow{\norm{\cdot}} x
        \iff
        \forall \varepsilon > 0 \;
        \exists n_0 \in \N \;
        \forall n \geq n_0 \;
        \norm{x - x_n} < \varepsilon
    $$
    a lze tedy zavést i nekonečné součty.
    Dále lze definovat cauchyovskost
    $$
        \{ x_n \} \text{ cauchyovská }
        \iff
        \forall \varepsilon > 0 \;
        \exists n_0 \in \N \;
        \forall m,n \geq n_0 \;
        \norm{x_m - x_n} < \varepsilon
    $$
    a \uu{úplnost $X$ v normě}:
    $$
        (X, \norm{\cdot}) \text{ je úplný v normě } \norm{\cdot}
        \Leftrightarrow{def}
        \left(
        \{ x_n \} \text{ cauchyovská }
        \implies
        \exists x \in X \; x_n \to x
        \right)\,.
    $$
    Je-li $(X, \norm{\cdot})$ úplný v normě $\norm{\cdot}$, nazývá se \uu{Banachův} prostor (B-prostor).
    
    \item Za známé dále považujeme, že pokud $\dim X < \infty$, pak všechny normy na něm jsou \uu{ekvivalentní}. Normy $\norm{\cdot}_1$ a $\norm{\cdot}_2$ nazveme ekvivalentní, pokud $\exists c_1, c_2 > 0$, že
    $c_1 \norm{x}_1 \leq \norm{x}_2 \leq c_2 \norm{x}_1 \; \forall x \in X$.
    
    \item Ekvivalentní normy zachovávají pojem konvergence ($x_n \xrightarrow{\norm{\cdot}_1} x \iff x_n \xrightarrow{\norm{\cdot}_2} x$) i~cauchyovskosti, a~tedy i~úplnosti. Speciálně, je-li konečnědimenzionální prostor $X$ úplný v $\norm{\cdot}$, je úplný i~ve všechn jiných možných normách na $X$.
    
    To neplatí v nekonečné dimenzi, např $\mathcal{C}([-1, 1])$ je úplný v maximové normě $\norm{f}_\infty \coloneqq \max_{[-1, 1]} |f(x)|$, ale není úplný v integrální normě $\norm{f}_1 \coloneqq \int_{-1}^1 |f|$. $\big(\arctg nx \to \frac{\pi}{2} \sgn x$ v $\norm{\cdot}_1\big)$
    
    \item \uu{Skalární součin na LP}: \hspace{1em} $\innerprod{\cdot}{\cdot}: X \times X \to \K \;$ (je-li $X$ nad $\C$, má tedy skalární součin komplexní hodnoty), je takové zobrazení, že $\forall x, y, z \in X$ platí:
    \begin{gather*}
        \innerprod{x}{y} = \conj{\innerprod{y}{x}},
        \\[3pt]
        \innerprod{x+y}{z} = \innerprod{x}{z} + \innerprod{y}{z},
        \\[3pt]
        \innerprod{x}{x} \geq 0,
        \text{, přičemž }
        \innerprod{x}{x} = 0
        \iff x = 0,
        \\[3pt]
        \innerprod{\alpha x}{y} =
        \alpha \, \innerprod{x}{y}
        \;\; \forall \alpha \in \K.
    \end{gather*}
    
    \item Prostor $\big(X, \innerprod{\cdot}{\cdot}\big)$ obdařený skalárním součinem se zve LP skalárním součinem, někdy též \uu{unitární prostor}.
    
    \item Snadno lze ukázat, že výraz $\norm{x} = \sqrt{\innerprod{x}{x}}$ má všechny vlastnosti normy, a tedy:
    
    $X$ je unitární $\implies$ $X$ je NLP (v~tzv. \uv{normě generované skalárním součinem})
    
    \item Pokud je $X$ \uu{úplný v normě generované skalárním součinem}, říká se mu \uu{Hilbertův prostor} (H-prostor), tedy:
    
    $X$ Hilbertův $\implies$ $X$ Banachův
    
    \item Na libovolném unitárním prostoru platí \uu{Cauchy-Schwarzova nerovnost}:
    $$
        \forall x,y \in X:
        \hspace{2em}
        |\innerprod{x}{y}| \leq \norm{x} \cdot \norm{y},
        \hspace{2em}
        \text{kde }
        \norm{x} = \sqrt{\innerprod{x}{x}}.
    $$
    
    \item $X$ unitární, řekneme, že $x,y \in X, \; x \neq 0, \; y \neq 0$ jsou \uu{kolmé} v $X$ (v odpovídajícím sk. součinu), pokud $\innerprod{x}{y} = 0$. Značíme $x \perp y$.
\end{itemize}

\Priklad{
    $L^2,\, L^2_\rho,\, \ell_2,\, W^{1,2},\, W^{k,2},\, W^{k,2}_\rho$ jsou Hilbertovy.
    \\
    $\mathcal{C}(\K),\, L^p$ pro $p \neq 2$ jsou Banachovy a nejsou Hilbertovy.
    \\ \\
    Existence normy (sk. součinu, příp. metriky) definuje na LP tzv. geometrické vlastnosti (vzdálenost, konvergence, pro sk. součin i kolmost).
}

\phantom{.}

Nyní připomeneme různé pojmy a vlastnosti související se \uu{zobrazeními} na vektorových prostorech.
\begin{enumerate}
    \item Buďte $X,Y$ LP (tj. nepotřebuji geometrii).
    \begin{itemize}
        \item \uu{operátor}: $T: X \to Y$
        \item \uu{funkcionál}: $T: X \to \K$
    \end{itemize}
    Každý funkcionál je i operátor. Budeme tedy BÚNO definovat další vlastnosti pro operátory.
    
    \item Operátor $T: X \to Y$ je
    \begin{itemize}
        \item \uu{lineární}: $T(ax + by) = a T(x) + b T(y) \hspace{2em} \forall x,y \in X \; \forall a,b \in \K$
        \item \uu{nelineární}: není lineární
    \end{itemize}
    
    \Poznamka z linearity $T$ plyne, že $T(0) = 0$ (volte $a=b=0$).
    
    \item $X$, $Y$ NLP, $T: X \to Y$ je
    \begin{itemize}
        \item \uu{omezený}: $\forall K>0 \; \exists C>0 \;\; \norm{x} \leq K \implies \norm{Tx} \leq C$ (zobrazuje se „omezená množina na omezené“),
        
        ekvivalentně $\exists C>0 \; \forall x \in X \;\; \norm{Tx} \leq C \norm{x} $.
        
        \item \uu{neomezený}: není omezený, tj. $\exists K>0 \; \forall C>0 \; \exists x_C \in X \;\; \norm{x_C} \leq K \And \norm{Tx} > C$
    \end{itemize}
    
    \item $X$, $Y$ Banachovy, $T: X \to Y$ je
    \begin{itemize}
        \item \uu{spojitý}: $x_n \to x \implies T x_n \to T x$ (tzv. „Heineova definice“)
        
        \item \uu{nespojitý}: není spojitý
    \end{itemize}
\end{enumerate}

Dále již budeme uvažovat pouze Banachovy (případně Hilbertovy) prostory, tj. vždy budeme mít úplnost.

\begin{definition}[Norma operátoru]
\label{1.Norma operatoru}
Mějme lineární operátor $T: X \to Y$. Definujeme číslo
$$
  \norm{T}_{\lin(X, Y)} \coloneqq \underset{\norm{x}_X \leq 1}{\sup} \norm{Tx}_Y.
$$
\end{definition}
Toto číslo může vyjít i nekonečno (např. po vhodně neomezený operátor). Pro lin. operátor však vidíme:
$$
  x \neq 0 \implies
  \norm{T\Big(\frac{x}{\norm{x}_X}\Big)}_Y \leq \norm{T}_{\lin(X, Y)}
  \hspace{2em}
  \text{($\norm{T}$ je supremem takových)}
$$
$$
    \norm{Tx}_Y \leq \norm{T}_{\lin(X, Y)} \cdot \norm{x}_X
$$
Platí i pro $\norm{T} = \infty$, $\forall x \neq 0$. Pokud $\norm{T} < \infty$, pak obě strany jsou konečné a  nerovnost platí $\forall x \in X$ (tj. včetně $x = 0$).
\\
\Poznamka připustili jsme $\norm{T} = \infty$, abychom měli tuto ekvivalenci:
\begin{lemma}[O charakterizaci omezenosti]
Pro lin. operátory máme:
$$
    T \text{ omezený}
    \iff
    \norm{T} < \infty
$$
a v tom případě má $\norm{T}$ vlastnosti normy (ověřte sami).
\end{lemma}
\begin{proof}
Implikace „$\Rightarrow$“: $T$ omezený: ($\operatorname{vol} K = 1: \; \exists C \; \forall \norm{x} \leq 1 \; \norm{Tx} \leq C$) $\implies$ $\norm{T} \leq C$.
\\[5pt]
Implikace „$\Leftarrow$“: $\norm{T} < \infty$: $\norm{Tx} \leq \norm{T} \cdot \norm{x} \; \forall x \in X$, tj. $\norm{x} \leq K \implies \norm{Tx} \leq K \norm{T}$.
\end{proof}

\begin{lemma}[O ekvivalenci spojitosti a omezenosti operátoru]
Buďte $X,Y$ Banachovy. Pokud je $T: X \to Y$ \uu{lineární} operátor, pak
$$
    \overset{(1)}{T \text{ omezený}}
    \iff
    \overset{(2)}{T \text{ spojitý}}
    \iff
    \overset{(3)}{\norm{T} < \infty.}
$$
\end{lemma}
\begin{proof}
Ekvivalenci $(1)$ a $(2)$ už jsme dokázali. Ukážeme:
\\[5pt]
$(3) \Rightarrow (2)$: Je-li $T$ omezený, pak $\norm{T (x_n - x)} \leq C \norm{x_n - x}, \; C = \norm{T}, \;$ z linearity $\norm{Tx_n - Tx} \leq C \norm{x_n - x}$. Pokud $x_n \to x$, pak odsud máme $Tx_n \to Tx$.
\\[5pt]
$(2) \Rightarrow (1)$: Ze spojitosti plyne mj., že pokud $x_n \to 0$, pak $Tx_n \to 0$ ($\forall x_n$). Proto $\forall \varepsilon$ (např. pro $\varepsilon = 1$) $\exists \delta>0$, že $\norm{x_n} < \delta \Rightarrow \norm{Tx_n} < \varepsilon = 1$. Buď nyní $\norm{x} < K$, pak $\norm{\frac{x}{K} \delta} < \delta \Rightarrow \norm{\frac{Tx}{K} \delta} < 1 \implies \norm{Tx} < \frac{K}{\delta} \eqqcolon C$.
\end{proof}

\begin{definition}[Prostor omezených lineárních operátorů]

$$
    \lin(X, Y)
    \coloneqq
    \{
        T: X \to Y, \;
        X,Y \text{ NLP}, \;
        T \text{ lineární omezený}
    \}.
$$
Lze ukázat, že $\lin(X, Y)$ sám o sobě je NLP s normou $\norm{\cdot}_{\lin(X,Y)}$ definovanou pomocí \ref{1.Norma operatoru}. Navíc, pokud $Y$ je Banachův, pak i $\lin(X, Y)$ je úplný v normě $\norm{\cdot}_{\lin(X,Y)}$, a tedy Banachův. Speciálně prostory operátorů $\lin(X, \R)$, $\lin(X, \C)$ jsou vždy Banachovy.
\\
Další používané značení:
\begin{gather*}
    \lin'(X) \coloneqq \lin(X, \K), \\
    \lin(X) \coloneqq \lin(X, X).
\end{gather*}
\end{definition}

\begin{theorem}[Vliv konečné a nekonečné dimenze]
Buď $T: X \to Y$, $X,Y$ Banachovy, $T$ lineární, \uu{$\dim T < \infty$}. Potom $T$ je omezený, a tedy i spojitý.
\end{theorem}
\begin{proof}
$$
    x \in X
    \implies
    x = \sum_{j=1}^n \alpha_j x_j, \text{ kde } n = \dim X, \text{ } \{ x_j \}_{j=1}^n \text{ báze } X
    \implies
    Tx = \sum_{j=1}^n \alpha_j T x_j
$$
$$
    \norm{Tx}
    \leq
    \sum_{j=1}^n |\alpha_j| \norm{Tx_j}
    \leq
    \underbrace{
        \underset{j=1,\dots n}{\max} \norm{Tx_j}
    }_{c}
    \; \cdot \;
    \sum_{j=1}^n |\alpha_j|
    =
    c \norm{x}_1
    \leq
    \Tilde{c} \norm{x}
$$
kde $\norm{a}_1 = \sum |a_j|$ je tzv. \textit{Manhattanská norma}, jedna z možných norem $X$, a poslední nerovnost plyne z ekvivalence všech norem pro $\dim X < \infty$.
\end{proof}

\Poznamka V konečné dimenzi jsou tedy všechny lineární operátory už spojité. Přirozená otázka: platí to i pro $\dim X = \infty$? Ne: $X, Y$ NLP, $\dim X = \infty$, pak $\exists T: X \to Y$ lineární a neomezený. (V případě $X,Y$ Banachovy tedy i nespojitý).
\\ \\
\Priklad{
    $X = \mathcal{C}^1([a,b])$ s normou $\norm{f} = \underset{[a,b]}{\sup} |f(x)|$. (V této normě není $X$ úplný, proč?)
    \\
    $Y = \mathcal{C}([a,b])$ s toutéž normou. (V ní je $Y$ Banachův, proč?)
    \\ \\
    Buď nyní $f_n(x) = \sin nx, \;\; f_n \in X, \;\; \norm{f_n} = 1$. \hspace{2em} $f'_n(x) = n \cos nx, \;\; f'_n \in Y, \;\; \norm{f'_n} = n$.
    \\
    Omezená množina se zobrazila na neomezenou $\implies$ operátor derivace je neomezený.
}
\\ \\ \\
\Cviceni{
    $\dim X = \infty$, $Y$ Banachův. Vezměte $\{ x_1, x_2, \dots \}$ LN spočetně nekonečnou množinu nenulových prvků v $X$. BÚNO $\norm{x_j} = 1$ (Jinak místo nich vezmeme $\frac{x_j}{\norm{x_j}}$.)
    
    Každou LN množinu lze podle tzv. \textit{Zornova lemmatu} (je ekvivalentní s axiomem výběru) doplnit na bázi LP. Doplňme ji tedy prvky $\{z_\alpha\}_{\alpha \in A}$, $A$ je tzv. \textit{indexová množina}.
    
    Potom dle vlastností báze $B \coloneqq \sequence{x_j}{j} \cup \{z_\alpha\}_{\alpha \in A}$ platí \hspace{0.5em}
    $\forall x \in X \; \exists n(x), m(x) \in \N \; \exists a_j, b_j \text{ skaláry}$, že
    $$
        x = \sum_{j=1}^{n(x)} a_j x_j + \sum_{k=1}^{m(x)} b_k z_k.
    $$
    Definujeme $\displaystyle Tx \coloneqq \sum_{j=1}^{n(x)} a_j T x_j + \sum_{k=1}^{m(x)} b_k T z_k$ pro takové $x \in X$.
    \\ \\
    Tím je definován $T$ na celém $X$, pokud definujeme $Tx_j$ a $Tz_\alpha$. Definujeme je takto:
    \begin{align*}
        T x_j &= j \;\; \forall j \in \N \\
        T z_\alpha &= 0 \;\; \forall z_\alpha, \; \alpha \in A
    \end{align*}
    Potom $T$ je lineární na $X$ (ověřte), přičemž $\norm{x_n} = 1$, ale $\norm{T x_n} = n \;\; \forall x_n$.
}


\pagebreak

