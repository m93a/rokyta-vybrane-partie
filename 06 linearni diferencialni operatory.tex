\section{Lineární diferenciální operátory}

V této kapitole zavedeme speciální případ lineárních neomezených operátorů, lineární diferenciální operátory, k nimž budeme generovat báze v polynomiálním tvaru.

\subsection{Výrazy v samoadjungovaném tvaru}

\begin{definition}[Lineární diferenciální výraz n-tého řádu]
Mějme $-\infty \leq a < b \leq \infty$, $y \in \mathcal{C}^n (a, b)$, 
Lineárním diferenciálním výrazem n-tého řádu nazveme výraz
\begin{equation}
    l(y) = \sum_{k=0}^{n} p_k (x) y^{(k)},
\end{equation}
kde $p_k(x) \in \mathcal{C}(a, b)$ a platí $p_n \not\equiv 0$ na $(a, b)$.
\end{definition}

\begin{remark}
    Ještě se nejedná o lineární diferenciální operátor. K definici lineárního diferenciálního operátoru potřebujeme navíc definiční obor.
\end{remark}

\begin{definition}[Lineární diferenciální operátor]
Pojmem lineární diferenciální operátor n-tého řádu $\map L$ myslíme příslušný lineární
diferenciální výraz $l(y)$ na určitém definičním oboru, tedy
\begin{align*}
    \map L = l \rvert_{\mathcal{D}(\map L)}
\end{align*}
\end{definition}

Budeme chtít aby $\map L$ byl hustě definovaný v Hilbertově prostoru $H$, tedy $\mathcal{D}(\map L) \neq H$, $\overline{\mathcal{D}(\map L)} = H$.
Typicky budeme mít $\mathcal{D}(\map L) = \left(H \cap \mathcal{C}^n (a,b)\right)$.

Víme, že symetrie je nutnou podmínkou k samoadjungovanosti. Dále budeme hledat další
nutné podmínky samoadjungovanosti, resp. symetrie a budeme pracovat na mnohem menším prostoru $\mathcal{C}^{\infty}_{cpt} (a,b)$, tedy nekonečněkrát diferencovatelné funkce
s kompaktním nosičem.
\begin{remark}
    Díky tomuto přístupu se při provádění per partes nemusíme zabývat okrajovými členy
    jelikož všechny funkce z uvažovaného prostoru $\mathcal{C}^{\infty}_{cpt} (a,b)$ (a jejich derivace) jsou na nějákém okolí krajních bodů nulové. Toto zjednodušení si můžeme dovolit protože hledáme nutné podmínky samoadjungovanosti, resp. symetrie. Pokud operátor nebude splňovat podmínky pro takto "hezké" funkce, nemůže je splňovat pro žádný jiný definičný obor.
\end{remark}

\begin{definition}[Adjungovaný výraz k $l(y)$]
Adjungovaný výraz k výrazu $l(y)$ značíme $l^*(y)$ a definujeme jej jako
\begin{equation}
    l^*(y) = \sum_{k=0}^{n} (-1)^{k} \left(\overline{p_k(y)} ~ y\right)^{(k)}
\end{equation}
\end{definition}

\begin{lemma}
K danému $l$ je $l^*$ jediný lineární diferenciální výraz pro který
\begin{align*}
    \left(l(y), z\right) = \left(y, l^*(z)\right)~ ~ ~ ~ ~ ~ ~ \forall y, z \in \mathcal{C}^{\infty}_{cpt} (a,b).
\end{align*}
\end{lemma}

\begin{proof}
Nejprve dokážeme pomocí per partes rovnost
\begin{align*}
\begin{split}
    \innerprod{l(y)}{z} =& \sum_{k=0}^{n} \int_a^b p_k(x) y^{(k)}(x) \overline{z(x)} =
    \sum_{k=0}^{n} (-1) \int_a^b \left(p_k(x) \overline{z(x)}\right)' y^{(k-1)} = \\
    =& \sum_{k=0}^{n} (-1)^{(k)} \int_a^b \left(p_k(x) \overline{z(x)}\right)^{(k)} y =
    \innerprod{y}{l^*(z)}.
\end{split}
\end{align*}
Poslední rovnost platí díky $\left(p_k(x) \overline{z(x)}\right) = \overline{\left(\overline{p_k(x)} z(x)\right)}$.

Jednoznačnost dokážeme sporem, předpokládejme že existují dva adjungované operátory $l^*$ a $\tilde{l}$ splňující rovnost. Pak
\begin{align*}
    \innerprod{l(y)}{z} = \innerprod{y}{l^*(z)} = \innerprod{y}{\tilde{l}(z)} ~ ~ ~ ~ ~ \forall y, z \in \mathcal{C}^{\infty}_{cpt} (a,b).
\end{align*}
Pak ale nutně z poslední rovnosti, která musí platit pro všechna $y$ z husté podmnožiny v $L^2$, dostáváme
\begin{align*}
    l^*(z) = \tilde{l}(z) ~ ~ ~ ~ ~ ~ \forall z \in \mathcal{C}^{\infty}_{cpt} (a,b),
\end{align*}
pokud si totiž zvolíme pevné $z$, pak ze spojitosti skalárního součinu musí být prvky v druhé složce skalárního součinu stejné. Na husté podmnožině tedy platí $l^*(z) = \tilde{l}(z)$ a z toho plyne
\begin{align*}
    l^* = \tilde{l}.
\end{align*}
\end{proof}
Další podmínkou kterou pro samoadjungovanost potřebujeme je $l = l^*$, což klade podmínku na tvar jednotlivých koeficientů $p_k$
\begin{align*}
    \sum_{k=0}^{n} p_k ~ y^{(k)} = \sum_{k=0}^{n} (-1)^{k} \left(\overline{p_k} ~ y\right)^{(k)} = \sum_{k=0}^{n} (-1)^{k} \sum_{j=0}^{k} {k \choose j} ~ \overline{p_k}^{(k-j)} y^{(j)}.
\end{align*}
Srovnejme koeficienty u $y^{(n)}$
\begin{align*}
    \begin{split}
        &p_n = (-1)^{n} \overline{p_n} \\
        &\text{$n$ sudé} : p_n = \overline{p_n} \implies \text{$p_n$ je reálný} \\
        &\text{$n$ liché} : p_n = - \overline{p_n} \implies p_n + \overline{p_n} = 2 Re (p_n) = 0 \implies \text{$p_n$ je ryze imaginární}
    \end{split}
\end{align*}
a dále postupujeme stejně. Nyní definujeme \uu{elementární diferenciální výrazy}, pomocí kterých lineární diferenciální výraz $l = l^*$ následně zapíšeme.
\begin{definition}[Elementární diferenciální výraz]
    Elementárním diferenciální výrazem nazveme LDV ve tvaru
    \begin{equation}
        \begin{split}
            E_{2k} &= (-1)^{k} \left(p ~ y^{(k)}\right)^{(k)} \\
            E_{2k-1} &= \frac{i}{2} \left[ \left(p ~ y^{(k-1)}\right)^{(k)} + \left(p ~ y^{(k)}\right)^{(k-1)} \right],
        \end{split}
    \end{equation}
kde $p$ je reálná funkce.
\end{definition}

\begin{theorem}[Čihák, str. 210]
    $l(y) = l^{*}(y) ~ \forall y \in \mathcal{C}^{\infty}_{cpt} \iff l(y)$ je konečnou lineární kombinací elementárních diferenciálních výrazů  $E_{2k}$ a $E_{2k-1}$.
\end{theorem}

\Priklad{
    Podívejme se, jak elementární diferenciální výraz výjde pro $k=1$ a $k=2$.
    \begin{equation*}
        \begin{split}
            E_1 &= \frac{i}{2}\left((p~y)' + p~y'\right) = \frac{i}{2} (p'~y + 2 p ~ y') = ip~y' + \frac{i}{2}p' ~ y. ~ \text{ Pro $p=1$ dostáváme }iy'. \\
            E_2 &= -(p~y')' \dots \text{ \uu{tzv. diferenciální výraz 2. řádu v samoadjungovaném tvaru}}.
        \end{split}
    \end{equation*}
}

\begin{remark}
    Pokud bychom pracovali ve vyšší dimenzi, bude $E_2$ odpovídat základnímu tvaru pro rovnici druhého řádu
    \begin{equation*}
        - \text{div} (p \nabla y).
    \end{equation*}
    Pro $p=1$ pak dostáváme Laplaceův operátor (při omezení na příslušný definiční obor)
    \begin{equation*}
        -\Delta y
    \end{equation*}
\end{remark}

\subsection{Ortogonální báze v $L^{2}_{\rho}$ složené z polynomů}
Budeme uvažovat prostor
\begin{align*}
    H = L^{2}_{\rho} (a, b) \equiv \left\{ f : (a, b) \to \C ; \int_a^b \rho \left|f\right|^2 < \infty, \text{ kde } \rho : (a, b) \to \R \text{ je tzv. \uu{váha} splňující } \rho > 0, \rho \in \mathcal{C}, \rho \in L^1 \right\}.
\end{align*}
\begin{remark}
    $\rho \in \mathcal{C}$ se někdy nevyžaduje.
\end{remark}
Lze ukázat, že $L^{2}_{\rho}(a, b)$ je Hilbertův prostor se skalárním součinem
\begin{align*}
    \innerprod{y}{z}_{2, \rho} \equiv \int_a^b \rho y \overline{z}
\end{align*}
a normou
\begin{align*}
    \norm{y}_{2, \rho} = \int_{a}^{b} \rho \left|y\right|^2.
\end{align*}
\begin{remark}
    Prostor $L^2_{\rho}$ 
\end{remark}