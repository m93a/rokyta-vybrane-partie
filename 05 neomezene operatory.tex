\section{Neomezené operátory}
Připomeňme, že jsme v úvodní kapitole ukázali charakterizaci spojitých lineárních operátorů: pro $\map T :X \mapsto Y$ lineární platí \begin{align*}
    \map T \text{ je omezený} \Longleftrightarrow \norm{\map T} < +\infty \Longleftrightarrow \map T \text{ je spojitý} \:.
\end{align*}
V této kapitole představíme některé výsledky o lineárních, ale neomezených a tedy nespojitých operátorech. Pro takové operátory už neplatí \begin{align*}
    x_n \rightarrow x \Rightarrow \map L x_n \rightarrow \map L x \:.
\end{align*}
Budeme pracovat pouze na Hilbertových prostorech, kde máme k dispozici vlastnosti skalárního součinu. Ukazuje se, že v případě neomezených operátorů je klíčová otázka definičního oboru nejen u operátoru samotného, ale i u k němu příslušnému adjungovanému.

Namísto $\map L'$ jako označení adjungovaného operátoru zde budeme používat označení $\map L^*$. Často budeme pracovat s prostory funkcí a symbol $\map L'$ by připomínal operaci derivování.
\subsection{Symetrie a samoadjungovanost}
\begin{definition}[Adjungovaný operátor]
Nechť $H$ je Hilbertův prostor, $\mathcal{D}(\map L) \subset H$ je lineární podprostor a $\map L : H \mapsto H$ je lineární operátor.
\begin{enumerate}
    \item Symbolem $\mathcal{D}(\map L^*)$ označujeme množinu všech $h \in H$ takových, pro které existuje právě jeden prvek $h^* \in H$ tak, že \begin{align*}
        (\map L x,y)_H = (x, h^*)_H \quad \text{pro všechna } x \in \mathcal{D}(\map L) \:.
    \end{align*}
    \item Je-li $\mathcal{D}(\map L^*)$ neprázdná množina, definujeme adjungovaný operátor $\map L^* : \mathcal{D}(\map L^*) \mapsto H$ předpisem \begin{align*}
        \map L^* y = h^* \:.
    \end{align*}
\end{enumerate}
\end{definition}

\begin{remark}
Je-li $\mathcal{D}(\map L^*) \neq \emptyset$, pak z definice ihned dostáváme \begin{align*}
    (\map L x,y)_H =(y, \map L^* y)_H \quad \text{pro všechna } x \in \mathcal{D}(\map L), y \in \mathcal{D}(\map L*) \:.
\end{align*}
Zatímco pro omezené (spojité) operátory je předchozí rovnost důsledkem Rieszovy-Fréchetovy věty (Věta \ref{4.Riesz-Frechet}), pro neomezené operátory je třeba tuto rovnost postulovat.
\end{remark}

Přirozeně zavádíme i pojem samoadjungovaného operátoru.

\begin{definition}[Samoadjungovaný operátor]
Operátor $\map L : \mathcal{D}(\map L^*) \mapsto H$ nazveme samoadjungovaný, jestliže \begin{enumerate}
    \item je množina $\mathcal{D}(\map L^*)$ neprázdná a $\mathcal{D}(\map L^*) = \mathcal{D}(\map L)$,
    \item platí $\map L^* = \map L$ na $\mathcal{D}(\map L^*)$.
\end{enumerate}
\end{definition}
\begin{remark}
Rovnost definičních oborů je zde velmi důležitá. Později uvidíme, že pro $\mathcal{D}(\map L^*) \neq \mathcal{D}(\map L)$ a $\map L^* = \map L$ na $\mathcal{D}(\map L^*) \cup \mathcal{D}(\map L)$ dostaneme úplně odlišné spektrální vlastnosti.
\end{remark}
Ihned z definice vidíme, že pokud $\map L^*$ existuje, je lineární.

První otázkou, která nás bude zajímat, je, za jakých podmínek je $\mathcal{D}(\map L^*)$ neprázdná množina?
\begin{theorem}[O existenci adjungovaného operátoru]
Nechť $H$ je Hilbertův prostor, $\map L$ je lineární operátor s definičním oborem $\mathcal{D}(\map L)$. Pak $\mathcal{D}(\map L^*)$ je neprázdná množina právě tehdy, když $\overline{ \mathcal{D}(\map L) }= H $.
\end{theorem}
\begin{proof}
Lze najít ve skriptech \textsc{Lukeš}.
\end{proof}
Přirozeně se hned ptáme, zda není definiční obor hustý v $H$ příliš malá množina a zda nelze volit přímo $ \mathcal{D}(\map L) = H $. Odpověď je překvapivá a negativní. Před její formulací ještě zavedeme pojem symetrického operátoru.

\begin{definition}[Symetrický operátor]
Nechť $H$ je Hilbertův prostor, $\map L : \mathcal{D}(\map L) \mapsto H$ je lineární operátor, $\overline{\mathcal{D}(\map L)} = H $. Řekneme, že $\map L$ je symetrický, jestliže \begin{align*}
    (\map L x, y)_H = (x, \map L y)_H \quad \text{pro všechna } x,y \in \mathcal{D}(\map L) \:.
\end{align*}
\end{definition}

Nyní si představíme překvapivý výsledek, který celou situaci komplikuje.

\begin{theorem}[O omezenosti symetrického operátoru na celém prostoru]
Nechť $H$ je Hilbertův prostor, $\map L$ je lineární a symetrický operátor definovaný na celém $H$. Pak je $\map L$ omezený.
\end{theorem}
\begin{proof}
Lze nalézt opět ve skriptech \textsc{Lukeš}.
\end{proof}

Pro samoadjungované operátory tedy nastává následující situace: $H$ je Hilbertův prostor, $\mathcal{D}(\map L)$ je lineární podprostor v $H$, přičemž  $\mathcal{D}(\map L) \neq H$, ale $\overline{\mathcal{D}(\map L)} = H$. Pak říkáme, že $\map L$ je hustě definovaný na prostoru $H$.

Terminologie používaná v literatuře se bohužel různí. Někteří autoři používají termínů \uv{hermitovský},\uv{samoadjungovaný} a \uv{symetrický} zcela odlišně nebo ve fyzikální literatuře dokonce tyto pojmy splývají. 

\begin{example}[Operátor derivování]
Zvolme $H = L^2 \left( (0,1) \right)$ a na něm definujme operátor $\map L f = f'$ s definičním oborem $\mathcal{D}(\map L) = C^1 \left( [0,1] \right)$. Víme, že definiční obor operátoru $\map L$ je hustý v $H$.  Zřejmě je lineární a neomezený.

Prozkoumejme možnou symetrii takového operátoru. Zřejmě je \begin{align*}
    (\map L f,g)_{L^2} =(f',g)_{L^2} = \int_0^1 f'(x) \overline{g}(x) \d x \:, \qquad (f,\map L g)_{L^2} = \int_0^1 f(x) \overline{g}'(x) \d x \:. 
\end{align*}
Aplikujeme-li metodu per-partes, dostaneme \begin{align*}
    \int_0^1 f'(x) \overline{g}(x) \d x = \left[ f \overline {g} \right]_0^1 - \int_0^1 f(x) \overline{g}'(x) \d x \:.
\end{align*}
Ani v případě, že se pomocí vhodných okrajových podmínek zbavíme krajních členů, nemůže být operátor $\map L$ nikdy symetrický, tedy ani samoadjungovaný.

Určíme nyní $\mathcal{D}(\map L^*)$, tedy vyšetříme množinu \begin{align*}
    \left \lbrace g \in C^1 \left( [0,1 ]\right) : \text{ existuje jednoznačné } h^* \in L^2\left( (0,1)\right) \text{ tak, že } (\map L f,g)_{L^2}=(f, h^*)_{L^2} \text{ pro každé } f \in C^1 \left( [0,1 ]\right)  \right \rbrace \:.
\end{align*}
Po funkci $h^*$ tedy požadujeme, aby splňovala \begin{align*}
    \left[ f \overline {g} \right]_0^1 - \int_0^1 f(x) \overline{g}'(x) \d x = \int_0^1 f(x) \overline{h^*}(x) \d x \quad \text{pro každé } f \in C^1\left( [0,1 ]\right) \:.
\end{align*}
Volbou speciálních funkcí $f$ zjistíme nutné podmínky. \begin{enumerate}
    \item DOKONČIT, obrázky jsou v \textrm{rokyta\_obrazky.nb}
    
    Volbou $f_\varepsilon(x)$ dostaneme podmínku $[\overline{g}]_0^1=0$.
    \item Volbou $f_\delta(x)$ dostaneme podmínku $g(0)=g(1)=0$.
\end{enumerate}
Odtud vidíme, že \begin{align*}
    \mathcal{D}(\map L^*) \subset \left \lbrace g \in C^1 \left( [0,1 ]\right) : g(0)=g(1)=0 \right \rbrace
\end{align*}
a požadovaná symetrie se redukuje na \begin{align*}
     -\int_0^1 f \overline{g}' \d x = \int_0^1 f \overline{h^*} \d x 
\end{align*}
Po převedení obou integrálů na pravou stranu a s použítím Du-Bois-Reymondova lemmatu dostáváme požadavek \begin{align*}
    h^* = -g' \quad \text{skoro všude} \:.
\end{align*}
(Protože hledáme řešení na třídě $C^1$ a tam patří funkce $g$, uvažujme pouze $h^* \in L^2\left( (0,1)\right) \cap C^1 \left(  [0,1] \right)$.)

Celkově jsme dostali \begin{align*}
    \mathcal{D}(\map L^*) = \left \lbrace g \in C^1 \left( (0,1)\right) \cap C^1 \left(  [0,1] \right) :  g(0)=g(1)=0 \text{ a zároveň } \map T^* g=-g^* \right \rbrace \:.
\end{align*}
Povšimněme si, že $\map L \neq \map L^* $ a navíc $\mathcal{D}(\map L^*) \subset \mathcal{D}(\map L)$, ale $\mathcal{D}(\map L^*) \subset \mathcal{D}(\map L)$.
\end{example}

\begin{example}[Modifikovaný operátor derivování]
Modifikací předchozího příkladu ověříme, že operátor \begin{align*}
    \map L f = i f' \quad \text{definovaný na } L^2 \left( (0,1) \right)
\end{align*}
by mohl splňovat podmínku symetrie.
Uvažujme tři možné definiční obory \begin{enumerate}
    \item $\mathcal{D}(\map T_1) = L^2 \left( (0,1) \right) \cap  C^1 \left( [0,1] \right)$,
    \item $\mathcal{D}(\map T_2) = L^2 \left( (0,1) \right) \cap  C^1 \left( [0,1] \right) \cap \{ f(x): f(0)=f(1)\}$,
    \item $\mathcal{D}(\map T_3) = L^2 \left( (0,1) \right) \cap  C^1 \left( [0,1] \right) \cap \{ f(x): f(0)=f(1)=0\}$
\end{enumerate}
a tři restrikce \begin{align*}
    \map T_1 = \map T\rvert_{\mathcal{D}(\map T_1)} \:, \qquad \map T_2 = \map T\rvert_{\mathcal{D}(\map T_2)} \:, \qquad \map T_3 = \map T\rvert_{\mathcal{D}(\map T_3)} \:.
\end{align*}
DOKONČIT
\end{example}

\begin{remark}
Připomeňme kvantověmechanický operátor hybnosti \begin{align*}
    \hat p = -i \hbar \dd{}{x} \quad \text{definovaný na } L^2(\R) \cap C^1(\R) \:.
\end{align*}
Modifikací předchozího příkladu snadno zjistíme, že operátor definovaný na takové množině je symetrický, ale není samoadjungovaný. Pokud bychom se snažili nalézt řešení rovnice \begin{align*}
    -i \hbar \dd{\psi}{x} = k \psi(x) \:,
\end{align*}
zjistili bychom, že vlastní funkce \begin{align*}
    \psi(x, k) = C e^{\frac{ik}{\hbar} x}
\end{align*}
nepatří do $L^2(\R)$. Z hlediska spektrální analýzy tedy nemá operátor hybnosti žádné vlastní vektory.
\end{remark}
\subsection{Spektrum neomezených operátorů}

V případě omezených operátorů hrály zásadní roli pro charakter spektra následující třídy operátorů: \begin{itemize}
    \item samoadjungované operátory, ty jsme zavedli i pro neomezené operátory, avšak komplikovanějším způsobem,
    \item kompaktní operátory, ovšem ty jsou podmnožinou omezených operátorů, proto je zde zavést nelze. Jejich roli částečně převezme třída uzavřených operátorů.
\end{itemize}

\begin{definition}[Uzavřený operátor]
Nechť $H$ je Hilbertův prostor, $\map L : \mathcal{D}(\map L) \mapsto H$ je lineární operátor hustě definovaný na $H$.  Řekneme, že $\map L$ je uzavřený, jestliže platí následující výrok: je-li $\sequence{x_n}{n} \subset \mathcal{D}(\map L) $ taková, že $x_n \rightarrow x$ a zároveň posloupnost $\sequence{\map L x_n}{n} \subset H$ konverguje k nějakému $g \in H$, pak $x \in \mathcal{D}(\map L)$ a $\map L x =g$.
\end{definition}

\begin{remark}
O uzavřeném operátoru také někdy říkáme, že má uzavřený graf. Uzavřenost se dá slovně formulovat zhruba takto: \uv{konverguje-li posloupnost vzorů a současně i posloupnost obrazů, pak musí limita obrazu být zobrazenou limitou vzoru.}
\end{remark}

Dále jsme u omezených operátorů studovali jejich surjektivnost a injektivnost (zda jsou operátory prosté a na). Tato otázka má jistě smysl i u neomezených operátorů. Konečně jsme takové studovali spojitost inverzního zobrazení. Touto otázkou se, celkem překvapivě, má smysl zabývat i v této kapitole. Klíčová pozorování jsou shrnuta v následující větě.

\begin{theorem}[O spektrálních vlastnostech neomezených operátorů]
Nechť $H$ je Hilbertův prostor, $\map L : \mathcal{D}(\map L) \mapsto H$ je lineární operátor hustě definovaný na $H$. Pak \begin{enumerate}
    \item Je-li $\overline{\mathcal{R}(\map L)} = H$, pak je $\map L$ prostý a na $\mathcal{R} (\map L)$.
    \item Je-li $\mathcal{R}(\map L) = H$, pak je $\map L$ prostý a na $H$, $\map L$ je samoadjungovaný, $\map L^{-1}$ je spojitý operátor.
    \item $\map L^{-1}$ je spojitý operátor právě tehdy, když $\map L$ je prostý a na $H$ a $\map L$ je uzavřený operátor.
\end{enumerate} 
\end{theorem}

\begin{definition}[Resolventa, spektrum operátoru]
Množina $ \left\lbrace x \in \C : \map L_\lambda \text{ je prostý, na, } \map L_\lambda^{-1} \text{ je spojitý} \right\rbrace$ se nazývá resolventa operátoru $\map L$ (značíme $\text{Res}\, \map L$). Spektrem operátoru $\map L$ nazýváme množinu $\sigma(\map L) := \C \setminus \text{Res}\, \map L$.
\end{definition}

\begin{remark}
Spektrum neomezeného operátoru může být jakákoli neomezená podmnožina $\C$, včetně celého $\C$.
\end{remark}

\begin{theorem}[O vlastnostech spektra neomezených operátorů]
Nechť $H$ je Hilbertův prostor a $\map L$ je lineární operátor na $H$. \begin{enumerate}
    \item Jestliže je $\map L$ uzavřený, pak $\sigma(\map L)$ je uzavřená množina v $\C$.
    \item Jestliže je $\map L$ uzavřený a symetrický, pak může nastat právě jedna z následujících situací: \begin{enumerate}
        \item $\sigma(\map L) = \C$,
        \item $\sigma(\map L) = \{ \lambda \in \C : \mathrm{Im}\, \lambda \geq 0 \}$,
         \item $\sigma(\map L) = \{ \lambda \in \C : \mathrm{Im} \, \lambda \leq 0 \}$,
          \item $\sigma(\map L)$ je uzavřená podmnožina $\R$.
    \end{enumerate}
    Pouze v posledním případě pak je $\map L$ samoadjungovaný, jinak je pouze symetrický.
    \item Je-li $\map L$ symetrický a má reálná vlastní čísla, pak jsou vlastní vektory příslušející různým vlastním číslům na sebe kolmé.
    \item Je-li $\map L$ uzavřený a samoadjungovaný, pak má pouze reálná vlastní čísla a vlastní vektory příslušející různým vlastním číslům jsou na sebe kolmé.
\end{enumerate}
\end{theorem}

\begin{remark}
\begin{enumerate}
    \item Druhá část předchozí věty ukazuje velkou odlišnost mezi samoadjungovaným a symetrickým operátorem. Neomezený symetrický operátor může obsahovat prvky z nebodového spektra, které nejsou reálné. U samoadjungovaného operátoru taková situace nemůže nastat.
    \item Vlastních čísel i vektorů může mít neomezený operátor obecně nespočetně mnoho. Partie matematiky zabývající se touto problematikou se nazývá spojitý funkcionální kalkulus, užívající obecnějšího integrálu namísto sumace při rozkladu prvků do vlastních vektorů.
    \item V teorii o neomezených operátorech nemáme apriori věty o úplnosti zadaného systému. V konkrétních případech je třeba jejich úplnost dokazovat případ od případu.
\end{enumerate}
\end{remark}