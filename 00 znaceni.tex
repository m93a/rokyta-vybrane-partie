
\begin{table}[h!]
\begin{tabular}{l|l}
$\conj{z}, \; \overline{u}, \; \overline{\Omega}$
& číslo komplexně sdružené k $z$, řešení blízké k $u$, uzávěr množiny $\Omega$
\\
$\mathcal{C}(\Omega), \; \mathcal{C}^k(\Omega)$
& prostor spojitých funkcí ($k$-krát spojitě diferencovatelných funkcí) na množině $\Omega$
\\
$L^p(\Omega), \; L^p_\rho(\Omega)$
& (vážený) Lebesgueův prostor na množině $\Omega$ (s váhou $\rho$)
\\
$W^{k,p}(\Omega), \; W^{k,p}_\rho(\Omega)$
& (vážené) Sobolevovy prostory na množině $\Omega$ (s váhou $\rho$)
\\
$\lin(X, Y), \; \lin(X)$
& prostor spojitých lineárních operátorů mezi prostory $X, Y$ (na prostoru $X$)
\\
$\comp(X, Y), \; \comp(X)$
& prostor kompaktních lineárních operátorů mezi prostory $X, Y$ (na prostoru $X$)
\\
$\mathcal{U}(x)$
& blízké okolí bodu $x$
\\
$\mathcal{D}(\map T)$
& definiční obor operátoru $\map T$
\\
$\mathcal{R}(\map T)$
& obor hodnot (range) operátoru $\map T$
\\
$\mathcal{M}^{m \times n}$
& množina všech $m\times n$ rozměrných matic nad $\R$ ($\C$)
\\
$\map T: X \to Y$
& operátor z $X$ do $Y$ (místo $\map T$ budeme psát $\map L$ pro neomezený a $\map K$ pro kompaktní operátor)
\\
$\Sp(\map T), \; \SpC, \; \SpP, \; \SpR$
& spektrum operátoru $\map T$; spojité, bodové a reziduální spektrum
\\
$\innerprod{f}{g}, \; \duality{f}{\map T}$
& skalární součin, dualita
\end{tabular}
\end{table}